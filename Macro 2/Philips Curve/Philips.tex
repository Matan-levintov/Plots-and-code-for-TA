% !TEX program = xelatex
\documentclass[usenames,dvipsnames]{beamer}
\usefonttheme{serif}
\usefonttheme{structuresmallcapsserif}
\usetheme{Warsaw}
\setbeamertemplate{headline}{} % This line removes the headline template
\usepackage{xcolor}
\beamertemplatenavigationsymbolsempty
\usepackage{tikz}
\usepackage{pgfplots}
\renewcommand{\qed}{\hfill\blacksquare}
\newcommand{\D}[1]{\Delta #1}
\renewcommand{\a}{\alpha}
\usepackage{float}
\setbeamertemplate{frametitle continuation}{%
    \ifnum\insertcontinuationcount>999999999 % this command tells the program when to start counting and also the count will be in numbers and not in roman letters
    \insertcontinuationcount
    \fi}
% ============================================================ %
% HEBREW support via polyglossia %
% ============================================================ %
\usepackage{polyglossia}
\defaultfontfeatures{Mapping=tex-text, Scale=MatchLowercase}
\setdefaultlanguage{hebrew}
\setotherlanguage{english}
\newfontfamily\hebrewfont[Script=Hebrew]{Arial}
% Use \begin{hebrew} block of text \end{hebrew} for paragraphs.
% Use \texthebrew{ } and \textenglish{ } for short texts.
% ============================================================ %
\title[]{עקומת פיליפס}
\author{מתן לבינטוב}
\institute[{{ אב"ג}}]{{ אוניברסיטת בן גוריון בנגב}}
% \date{}
\usepackage{bidi}
\begin{document}
\begin{RTL}
\begin{frame}
\titlepage
\end{frame}
\begin{frame}
    \frametitle{נושאים}
    \tableofcontents

    

\end{frame}

\section{עקומת פיליפס}
\begin{frame}[allowframebreaks]
    \frametitle{עקומת פיליפס}
    \begin{block}{עקומת פיליפס}
        עקומת פיליפס מתארת את הקשר השלילי בין אינפלציה לאבטלה
    \end{block}

    \begin{exampleblock}{התיקון של פרידמן ופלפס}
        קיים קשר שלילי בין אינפלציה ואבטלה אך רק בט''ק ורק במידה והאינפלציה היא בלתי צפויה
    \end{exampleblock}
    \begin{block}{הסבר}
        בטווח הקצר השכר קשיח ועליית מחירים מקטינה את השכר הריאלי בכך מגדילה את התעסוקה / מקטינה אבטלה
        $$p \uparrow \implies \frac{w}{p} \downarrow \implies L^d \uparrow \implies u \downarrow$$
    \end{block}
    
    \framebreak

    \begin{block}{ציפיות לאינפלציה לעומת אינפלציה בפועל}
        ציפיות לאינפלציה משפיעות על השכר הנומינלי, לעומת אינפלציה בפועל שמשפיע על השכר הריאלי,
        $$\frac{w (1 + \pi^e)}{p (1 + \pi)}$$
        $$y = y_p \quad u = u_n \quad \left(\dfrac{w}{p}\right)^{\text{חדש}} = \left(\dfrac{w}{p}\right)^{\text{ישן}} \quad \pi = \pi^e$$
        $$y  > y_p \quad u < u_n \quad \left(\dfrac{w}{p}\right)^{\text{חדש}} < \left(\dfrac{w}{p}\right)^{\text{ישן}} \quad \pi > \pi^e$$
        $$y < y_p \quad u > u_n \quad \left(\dfrac{w}{p}\right)^{\text{חדש}} > \left(\dfrac{w}{p}\right)^{\text{ישן}} \quad \pi < \pi^e$$
    \end{block}

    אנחנו מניחים שבטווח ארוך, הציפיות יתאימו את עצמם לאינפלציה ולכן לא יהיו סטיות בשכר ובאבטלה.

\end{frame}

\section{סוגי ציפיות}
\begin{frame}
    \frametitle{סוגי ציפיות}
    \begin{itemize}
        \item ציפיות רציונליות - הציבור מתוחכם, לוקח בחשבון את כל המידע שיש ויודע את התוצאה הסופית $\pi = \pi ^ e$
        \item ציפיות נאיביות (מקרה פרטי של אדפטיבי) - מה שהיה זה מה שיהיה $\pi_t = \pi_{t-1}$
    \end{itemize}
    

\end{frame}
\section{עקומת פיליפס ופונקציית ההפסד של המדינאי}
\begin{frame}[allowframebreaks]
    \frametitle{עקומת פיליפס ופונקציית ההפסד של המדינאי}
    $$u = u_n - \alpha(\pi - \pi^e)$$
    \begin{center}
        \begin{tikzpicture}
            \begin{axis}[
                axis lines = left,
                xlabel = \( u_t \),
                ylabel = {\( \pi_t \)},
                xmin=0, xmax=10,
                ymin=0, ymax=10,
                clip=false,
                xtick=\empty,
                ytick=\empty
            ]
            
            % Add the plot line
            \addplot [thick] coordinates {(0,8) (8,0)};
            
            % Dotted lines
            \draw [dotted] (axis cs:5,3) -- (axis cs:5,0) node[below] {\( u^n \)};
            \draw [dotted] (axis cs:5,3) -- (axis cs:0,3) node[left] {\( \pi_t^e \)};
            
            \end{axis}
            \end{tikzpicture}
            
    \end{center}
    
    \framebreak
    $$L (\pi^+,u^+)$$
    \begin{center}
        \begin{tikzpicture}
            \begin{axis}[
                axis lines = left,
                xlabel = \( u_t \),
                ylabel = {\( \pi_t \)},
                xmin=0, xmax=10,
                ymin=0, ymax=10,
                clip=false,
                xtick=\empty,
                ytick=\empty
            ]
            
            % Origin point
            % \node at (axis cs:0,0) {(0,0)};
            
            % Add the curved lines, manually adjusted for a closer fit
            \addplot [blue, thick, domain=0:8.5, samples=100] {8-x^2/9};
            \addplot [blue, thick, domain=0:6.93, samples=100] {6-x^2/8};
            \addplot [blue, thick, domain=0:5.3, samples=100] {4-x^2/7};
            
            % Labels for the curves
            \node at (axis cs:8.5,2) {\( L_1 \)};
            \node at (axis cs:6.5,2) {\( L_2 \)};
            \node at (axis cs:4.2,2) {\( L_3 \)};
            
            \end{axis}
            \end{tikzpicture}
            
    \end{center}

\end{frame}
\section{גישות של המדינאי}
\begin{frame}[allowframebreaks]
    \frametitle{גישות של המדינאי}
    \begin{block}{גישה של שיקול דעת}
        המדינאי מנסה להביא למינימום את פונקציית ההפסד שלו (בכל תקופה)
        $$\min L = f \ (u,\pi)$$
        $$s.t \quad u = u_n - \alpha(\pi - \pi^e)$$
    \end{block}
    \begin{exampleblock}{סדר פעולות לפיתרון}
        \begin{enumerate}
            \item מציבים את עקומת פיליפס לפונקציית ההפסד
            \item גוזרים ומשווים ל0
            \item מבודדים את האינפלציה
            \item הצבת האינפלציה  בעקומת פיליפס בשביל למצוא את הפיתרון
        \end{enumerate}
    \end{exampleblock}
    \begin{alertblock}{הערה}
        במידה והציפיות רציונליות הציבור יודע לחשב את התוצאה הסופית ולכן פשוט נציב $\pi = \pi^e$
    \end{alertblock}
    \framebreak
    \begin{block}{כללים ברורים}
        המדינאי מכריז מראש כיצד הוא יפעל, 
        $$\pi = \pi^e \quad u = u_n$$
        מכאן, סדר הפעולות של הפיתרון הוא, להציב $u = u_n$ ולמצוא את האינפלציה שתביא למינימום של פונקציית ההפסד.
    \end{block}
\end{frame}

\section{רתיעה מאינפלציה}
\begin{frame}
    \frametitle{רתיעה מאינפלציה}
    \begin{block}{דוגמה לפונקציית עלות}
        $$L = u^2 + \gamma \pi^2$$
        $\gamma$ מבטאת את הרתיעה של המדינאי מאינפלציה
    \end{block}
    \begin{alertblock}{מקרי קיצון}
        \begin{enumerate}
            \item $\gamma = 0$ המדינאי אדיש לאינפלציה (מפחד רק מאבטלה)
            \item $\gamma = \infty$ המדינאי רק מפחד מאינפלציה (אדיש מאבטלה)
        \end{enumerate}
    \end{alertblock}
\end{frame}
\end{RTL}
\end{document}