% !TEX program = xelatex
\documentclass[usenames,dvipsnames]{beamer}
\usefonttheme{serif}
\usefonttheme{structuresmallcapsserif}
\usetheme{Copenhagen}
\usepackage{xcolor}
\beamertemplatenavigationsymbolsempty
\usepackage{tikz}
\usepackage{pgfplots}
\pgfplotsset{compat=1.18}
\renewcommand{\qed}{\hfill\blacksquare}
\newcommand{\D}[1]{\Delta #1}
\renewcommand{\a}{\alpha}
\usepackage{float}

% ============================================================ %
% HEBREW support via polyglossia %
% ============================================================ %
\usepackage{polyglossia}
\defaultfontfeatures{Mapping=tex-text, Scale=MatchLowercase}
\setdefaultlanguage{hebrew}
\setotherlanguage{english}
\newfontfamily\hebrewfont[Script=Hebrew]{Arial}
\usepackage{bidi}
% ============================================================ %

\title[מלכודת נזילות]{{תרגול 2 - מלכודת נזילות}}
\author{{ מתן לבינטוב}}
\institute[{{ אב"ג}}]{{ אוניברסיטת בן גוריון בנגב}}
\date{}

\begin{document}
\begin{RTL}
\begin{frame}
\titlepage
\end{frame}

\begin{frame}
    \frametitle{מצבי קיצון $LM$}
    \begin{block}{תורת כמות הכסף $h=0$}
        משמע הביקוש לכסף אינו תלוי בריבית, מכך שהרחבה פיסקלית אינה יעילה ויהיה קיזוז מלא ונקבל נקודה  עם אותו תוצר אבל עם ריבית גבוהה יותר
        $$LM : i = \frac{1}{h} \cdot \left[ky - \frac{M}{P}\right]$$
    \end{block}

    \begin{block}{מלכודת נזילות $h=\infty$}
        הביקוש לכסף הוא אינסופי, כלומר הציבור יהיה מוכן להחזיק כל כמות של כסף, מכך שהרחבה פיזקלית היא מאוד יעילה משום שאין קיזוז כתוצאה מעליית ריבית (היות ועליית הריבית אינה מתחרשת)
        $$LM : i = \frac{1}{h} \cdot \left[ky - \frac{M}{P}\right]$$   
    \end{block}
\end{frame}

\begin{frame}
    \frametitle{מצבי קיצון $LM$, $h=0$}

    \begin{center}
            \begin{tikzpicture}
                    % Axes
                    \draw[->] (0,0) -- (6,0) node[right] {$Y$};
                    \draw[->] (0,0) -- (0,6) node[above] {$i$};
                    
                    % IS curve (linear)
                    \draw[thick, blue] (0.5,5.5) -- (5.5,0.5) node[right] {$IS$};
                    % LM curve (linear)
                    \draw[thick, red] (2,0) -- (2,6) node[right] {$LM$};
                    % Liquidity trap horizontal line
                    % \draw[dashed] (0,0) -- (6,0) node[right] {Liquidity Trap};
                    % Equilibrium point
                    % \fill[black] (3,3) circle (2pt) node[above] {Equilibrium};
                \end{tikzpicture}
    \end{center}
\end{frame}

\begin{frame}
    \frametitle{מצבי קיצון $LM$, $h=\infty$}

    \begin{center}
            \begin{tikzpicture}
                    % Axes
                    \draw[->] (0,0) -- (6,0) node[right] {$Y$};
                    \draw[->] (0,0) -- (0,6) node[above] {$i$};
                    
                    % IS curve (linear)
                    \draw[thick, blue] (0.5,5.5) -- (4,-0.5) node[right] {$IS$};
                    % LM curve (linear)
                    \draw[thick, red] (0,0) -- (5.7,0) node[above] {$LM$};
                    % Liquidity trap horizontal line
                    % \draw[dashed] (0,0) -- (6,0) node[right] {Liquidity Trap};
                    % Equilibrium point
                    % \fill[black] (3,3) circle (2pt) node[above] {Equilibrium};
                \end{tikzpicture}
    \end{center}
\end{frame}

\begin{frame}
    \frametitle{מלכודת נזילות}
    \begin{block}{מלכודת נזילות}
        משק שנמצא בשפל עמוק, עם מחסור בביקושים וריבית במשק אפסית $(i=0)$ \\
        במצג רגיל הרחבה מוניטרית : \\
        $\text{ביקוש לאגח} \uparrow \implies \text{מחיר האגח} \uparrow \implies \text{תשואת האג''ח }\downarrow \implies i\downarrow$ \\
        אולם, במלכודת נזילות הרחבה מוניטרית אינה מעלה את הביקוש לאג''ח מחשש לעליית ריבית ובכך פגיעה בערך האג''ח או משום שאין פיצוי על הסיכון באג''ח. \\
        במילים אחרות הציבור פשוט סופג את הכסף בלי שינוי במחירים או תשואות. \\
        מכך ניתן להסיק שהפיתרון במלכודת נזילות הוא הרחבה פיסקלית מכיוון שאין קיזוז והעלייה בתוצר תהיה בגודל המכפיל הפיסקלי הפשוט (בגודל היסט האופקי $\a$)
    \end{block}

    \begin{alertblock}{תזכרו}
        $$\a > \beta$$
    \end{alertblock}
    

\end{frame}

\begin{frame}
    \frametitle{מלכודת נזילות}

    \begin{center}
        \begin{tikzpicture}
                % Axes
                \draw[->] (0,0) -- (6,0) node[right] {$Y$};
                \draw[->] (0,0) -- (0,6) node[above] {$i$};
                
                % IS curve (linear)
                \draw[thick, blue] (0.5,2) -- (3,-1) node[right] {$IS_1$};
                \draw[thick, blue] (1.5,2) -- (4,-1) node[right] {$IS_2$};
                \draw[thick, blue] (3.5,2) -- (5.75,-1) node[right] {$IS_3$};


                % LM curve (linear)
                \draw[thick, red] (0,0) -- (5,0) node[above]{};
                \draw[thick, red] (5,0) -- (7,3) node[above]{$LM$};

                % Liquidity trap horizontal line
                % \draw[dashed] (0,0) -- (6,0) node[right] {Liquidity Trap};
                % Equilibrium point
                % \fill[black] (3,3) circle (2pt) node[above] {Equilibrium};
            \end{tikzpicture}
    \end{center} 



\end{frame}

\begin{frame}
    \frametitle{מלכודת נזילות עם ירידת מחירים}
    \begin{itemize}
        \item כשאנחנו נמצאים באבטלה אנחנו מצפים שמחירים ירדו
        \item הסכנה בירידת מחירים היא ציפייה לכך שירידת המחירים תמשיך
        \item ציפיות אלו מקפיאות השקעות והציבור כל הזמן מחכה לירידת מחירים הבאה
    \end{itemize}

    

\end{frame}

\begin{frame}
    \frametitle{תזכורת - ביקוש להשקעות}
    הביקוש להשקעות $I$ מושפע מהריבית הריאלית :
    \begin{equation*}
        r^e = i - \pi^e
    \end{equation*}
    \begin{equation*}
        I = I_0 - br = I_0 - b(i - \pi^e) = \textcolor{red}{I_0 - bi + b\pi^e}
    \end{equation*}
    כאשר הציפיות לאינפלציה חיויביות : \\
    \begin{equation*}
        \pi^e > 0 \implies  r^e  < i \implies I \uparrow
    \end{equation*}
    כאשר הציפיות לאינפלציה שליליות : \\
    \begin{equation*}
        \pi^e < 0 \implies  r^e  > i \implies I \downarrow
    \end{equation*}
\end{frame}

\begin{frame}
    \frametitle{דוגמה - ירידה בביקושים}
    \begin{center}
        \begin{tikzpicture}[scale=.95]
                % Axes
                \draw[->] (0,0) -- (6,0) node[right] {$Y$};
                \draw[->] (0,0) -- (0,6) node[above] {$i$};
                
                % IS curve (linear)
                \draw[thick, blue] (0.5,5.5) -- (5.5,0.5) node[right] {$IS_1$};
                \draw[thick, blue] (0.5,4.5) -- (5.5,-0.5) node[right] {$IS_2(A_0 \downarrow)$};
                \draw[thick, blue] (0.5,3.5) -- (5.5,-1.5) node[right] {$IS_3(\pi^e < 0 )$};

                % LM curve (linear)
                \draw[thick, red] (0.5,0.5) -- (5.5,5.5) node[left] {$LM_1$ \ \ };
                \draw[thick, red] (1.5,0.5) -- (6.5,5.5) node[above] {$LM_2 (P \downarrow)$};
                \draw[thick, red] (2.5,0.5) -- (7.5,5.5) node[right] {$LM_3 (P \downarrow)$};


                %Y_P line
                \draw[thick,black] (3,0) -- (3,6) node[right] {$Y_p$};
            \end{tikzpicture}
    \end{center}
    

\end{frame}

\begin{frame}
    \frametitle{מסקנות}
    \begin{exampleblock}{מסקנות}
        יש לבצע מדיניות אקטיבית שתמנע ירידת מחירים. יש לבצע אותה באופן מידי כאשר מזהים מיתון על מנת
למנוע את ירידת המחירים. \textbf{ניתן לבצע גם מדיניות מוניטרית מרחיבה וגם מדיניות פיסקלית מרחיבה}.
    \end{exampleblock}
    


\end{frame}


\end{RTL}
\end{document}