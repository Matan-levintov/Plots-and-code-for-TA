% !TEX program = xelatex
\documentclass[usenames,dvipsnames]{beamer}
\usefonttheme{serif}
\usefonttheme{structuresmallcapsserif}
\usetheme{Warsaw}
\setbeamertemplate{headline}{} % This line removes the headline template
\usepackage{xcolor}
\beamertemplatenavigationsymbolsempty
\usepackage{tikz}
\usepackage{pgfplots}
\renewcommand{\qed}{\hfill\blacksquare}
\newcommand{\D}[1]{\Delta #1}
\renewcommand{\a}{\alpha}
\usepackage{float}
\setbeamertemplate{frametitle continuation}{%
    \ifnum\insertcontinuationcount>999999999 % this command tells the program when to start counting and also the count will be in numbers and not in roman letters
    \insertcontinuationcount
    \fi}
% ============================================================ %
% HEBREW support via polyglossia %
% ============================================================ %
\usepackage{polyglossia}
\defaultfontfeatures{Mapping=tex-text, Scale=MatchLowercase}
\setdefaultlanguage{hebrew}
\setotherlanguage{english}
\newfontfamily\hebrewfont[Script=Hebrew]{Arial}
% Use \begin{hebrew} block of text \end{hebrew} for paragraphs.
% Use \texthebrew{ } and \textenglish{ } for short texts.
% ============================================================ %
\title[]{מס אינפלציה וסניורז'}
\author{מתן לבינטוב}
\institute[{{ אב"ג}}]{{ אוניברסיטת בן גוריון בנגב}}
\date{}
\usepackage{bidi}
\begin{document}
\begin{RTL}
\begin{frame}
\titlepage
\end{frame}
\begin{frame}
    \frametitle{נושאים}
    \tableofcontents
    

\end{frame}
\section{דרכי מימון של הממשלה והשפעת על החוב הציבורי}
\begin{frame}
    \frametitle{דרכי מימון של הממשלה והשפעת על החוב הציבורי}
    \begin{itemize}
        \item מימון על ידי העלת מיסים (חוב ציבורי לא גדל)
        \item מימון על ידי הלוואה מהציבור (חוב ציבורי גדל)
        \item מימון על ידי הדפסת כסף (חוב ציבורי לא גדל)
    \end{itemize}
    

\end{frame}

\section{סניורז  $SE$ }
\begin{frame}
    \frametitle{סניורז $SE$}
    \begin{block}{סניורז' - הכנסות הממשלה מהדפסת כסף}
        $$SE = \frac{M_t - M_{t-1}}{P_t} = \frac{\Delta M}{P} = \frac{\Delta M}{M} \cdot \frac{M}{P} = m \cdot \frac{M}{P}$$
        לפי תורת כמות הכסף,
        $$SE = m \cdot \frac{M}{P} = (\pi + \widehat{Y} - \widehat V) \cdot \frac{M}{P}$$
    \end{block}
    \begin{itemize}
        \item $\widehat{V} \cdot \dfrac{M}{P}$ - אנחנו מניחים ששינוי במהירות המחזור יכול להיות רק בטווח הקצר, ולכן בטווח ארוך $\widehat{V} = 0$
        \item $\widehat{Y}\cdot \dfrac{M}{P}$ - מס צמיחה
        \item $\pi \cdot \dfrac{M}{P}$ - מס אינפלציה
    \end{itemize}
    

\end{frame}

\section{מס אינפלציה}
\begin{frame}[allowframebreaks]
    \frametitle{מס אינפלציה}
    הממשלה שולטת ברמת האינפלציה היות והיא קובעת כמה כסף להדפיס אך היא לא שולטת לחלוטין במס האינפלציה,

    \begin{itemize}
        \item מצד אחד, $$\pi \uparrow \implies IT \uparrow$$
        \item מצד שני, $$\pi \uparrow \implies i \uparrow \implies \frac{M^d}{P} \downarrow \implies IT \downarrow$$
    \end{itemize}

    \framebreak
    \begin{center}
        \begin{tikzpicture}[scale=1.25]
            % Axis
            \draw[->] (0,0) -- (5,0) node[right] {$\pi$};
            \draw[->] (0,0) -- (0,4) node[above] {$\pi \cdot \dfrac{M}{P}$};
          
            % Arc
            \draw[thick] (0,0) to[bend left=60] (4.5,0);
          
            % Points
            \filldraw [blue] (0.5,0.55) circle (2pt) node[align=center, below] {\\ $A$}
                             (2.25,1.15) circle (2pt) node[align=center, above] {$C$ \\ }
                             (4,0.6) circle (2pt) node[align=center, below] {\\ $B$};
          
            % Dotted lines
            % \draw[dotted] (1,1) -- (1,0);
            % \draw[dotted] (4,1) -- (4,0);
          \end{tikzpicture}    
    \end{center}
    \begin{itemize}
        \item $A$ - גמישות קטנה מ1, עלייה באינפלציה תעלה את מס האינפלציה
        \item $C$ - גמישות שווה ל1, עלייה באינפלציה לא תשנה את מס האינפלציה
        \item $B$ - גמישות גדולה מ1 , עלייה באינפלציה תוריד את מס האינפלציה
    \end{itemize}
   

    

\end{frame}

\section{קשר בין מס אינפלציה למהירות המחזור}
\begin{frame}
    \frametitle{קשר בין מס אינפלציה למהירות המחזור}
    \begin{align*}
        IT &= \pi \cdot \frac{M}{P} \\
        MV &= PY \iff \frac{M}{P} = \frac{Y}{V} \implies \\
        IT &= \pi \cdot \frac{Y}{V} \implies
    \end{align*}

    \begin{block}{משקל מס האינפלציה בתוצר}
        \begin{equation*}
            \frac{IT}{Y} = \frac{\pi}{V}
        \end{equation*}    
        \textbf{שימו לב שככל שמהירות המחזור גדולה יותר, כך המשקל של מס האינפלציה מתוך התוצר נמוך יותר}
    \end{block}
    
    
    

\end{frame}

\section{מס אינפלציה בטווח הקצר}
\begin{frame}[allowframebreaks]
    \frametitle{מס אינפלציה בטווח הקצר}
    \begin{block}{מי מפסיוד ומי מרוויח מציפיות לא מדויקות}
        הריבית הנומינלית בכל רגע נתון נקבעת כך, 
        $$i = r^e + \pi^e$$
        הריבית הריאלית בפועל (בדיעבד) היא,
        $$r = i - \pi = (r^e + \pi^e) - \pi$$
        מכאן, שנוצר פער בין הריבית הריאלית בפועל לריבית הריאלית הצפויה
        $$r - r^e = \pi^e - \pi$$ 
    \end{block}
    \framebreak
    \begin{block}{מי מפסיד ומי מרוויח מציפיות לא מדויקות}
    כעת נשים לב מה קורה בשני המקרים,
    \begin{enumerate}
        \item אם האינפלציה גבוהה מהאינפלציה הצפויה, הריבית הריאלית נשחקת ביחס לצפויה, הלווה (ממשלה) מרוויחה והמלווה (ציבור) מפסיד
        $$\pi > \pi^e \implies r < r^e$$
        \item אם האינפלציה נמוכה מהאינפלציה הצפויה, הריבית הריאלית גבוהה ביחס לצפויה, הלווה (ממשלה) מפסידה המלווה (ציבור) מרוויח 
        $$\pi < \pi^e \implies r > r^e$$
    \end{enumerate}
    \end{block}

\end{frame}


\section {אפקט טנזי אוליברה}
\begin{frame}
    \frametitle{אפקט טנזי אוליברה}
    \begin{block}{אפקט טנזי אוליברה}
        שני כלכלנים אשר בחנו את ההשפעה על האינפלציה על גביית מיסים, החוקרים גילו שהנזק שנוצר משחיקת
הערך הריאלי של גביית המיסים עולה על התועלת של הממשלה ממס אינפלציה. 
    \end{block}
    

\end{frame}
\end{RTL}
\end{document}