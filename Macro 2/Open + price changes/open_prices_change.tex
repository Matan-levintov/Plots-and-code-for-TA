% !TEX program = xelatex
\documentclass[usenames,dvipsnames]{beamer}
\usefonttheme{serif}
\usefonttheme{structuresmallcapsserif}
\usetheme{Warsaw}
\setbeamertemplate{headline}{} % This line removes the headline template
\usepackage{xcolor}
\beamertemplatenavigationsymbolsempty
\usepackage{tikz}
\usepackage{pgfplots}
\renewcommand{\qed}{\hfill\blacksquare}
\newcommand{\D}[1]{\Delta #1}
\renewcommand{\a}{\alpha}
\usepackage[utf8]{inputenc} 
\usepackage{float}
\usepackage{amsmath}
\setbeamertemplate{frametitle continuation}{%
    \ifnum\insertcontinuationcount>999999999 % this command tells the program when to start counting and also the count will be in numbers and not in roman letters
    \insertcontinuationcount
    \fi}

\usepackage{graphicx}
% ============================================================ %
% HEBREW support via polyglossia %
% ============================================================ %
\usepackage{polyglossia}
\defaultfontfeatures{Mapping=tex-text, Scale=MatchLowercase}
\setdefaultlanguage{hebrew}
\setotherlanguage{english}
\newfontfamily\hebrewfont[Script=Hebrew]{Arial}
% Use \begin{hebrew} block of text \end{hebrew} for paragraphs.
% Use \texthebrew{ } and \textenglish{ } for short texts.
% ============================================================ %
\title[]{משק פתוח במחירים משתנים}
\author{מתן לבינטוב}
\institute[{{ אב"ג}}]{{ אוניברסיטת בן גוריון בנגב}}
\date{}
\usepackage{bidi}
\begin{document}
\begin{RTL}
\begin{frame}
\titlepage
\end{frame}
\begin{frame}
    \frametitle{נושאים}
    \tableofcontents
    

\end{frame}


\begin{frame}
    \frametitle{שלושי טווחי זמן}
    אנו נבחין בין 3 טווחי זמן
    \begin{itemize}
        \item טווח מיידי - $\bar{P} , \bar{W}$
        \item טווח קצר - $P \uparrow \downarrow, \bar{W}$
        \item טווח ארוך - $P \uparrow \downarrow, W \uparrow \downarrow$
    \end{itemize}

    

\end{frame}

\section{סדר פעולות לפתרון בשע''ח קבוע}
\begin{frame}[allowframebreaks]
    \frametitle{סדר פעולות לפתרון בשע''ח קבוע}
    \begin{block}{שלב $I$}
        שרטוט של מצב המוצא כולל התייחסות לתנועות הון. נקודה $A$
        \begin{itemize}
            \item משק סגור לתנועות הון - $BP$ קשיחה לחלוטין
            \item משק פתוח לתנועות הון מושלמות - $BP$ גמישה לחלוטין
            \item משק פתוח לתנועות הון - שני אפשרויות
            \begin{enumerate}
                \item $BP$ גמישה יותר מ $LM$
                \item $BP$ קשיחה יותר מ $LM$
            \end{enumerate}
        \end{itemize}
    \end{block}

    \begin{block}{שלב $II$}
        שרטוט של השינוי שקרה במשק, נקודה $B$
        \begin{itemize}
            \item $IS$ - $C_0 \uparrow ,cT_0 \downarrow ,I_0 \uparrow ,G_0 \uparrow ,NX_0 \uparrow ,E \uparrow ,P^{\ *} \uparrow$ \\ הרחבה ימינה ולמעלה
            \item $LM$ - $M_0 \downarrow , M^s \uparrow$ \\ הרחבה ימינה ולמטה
            \item $BP$ - $NX_0 \uparrow , E \uparrow , P^{\ *} \uparrow , TR_0 \uparrow, CF_0 \uparrow, i^{\ *} \downarrow$ \\ הרחבה ימינה ולמטה
        \end{itemize}
    \end{block}
    
    \begin{block}{שלב $III$}
        ש''מ של טווח מיידי - נבחן איפה נקודת החיתוך $IS = BP$, נקודה $C$
        \begin{itemize}
            \item חיתוך $IS = LM$ נמצא שמאלה / למעלה ביחס לעקומת $BP$ - עודף במאזן התשלומים / עודף היצע / $BP > 0$. לכן הבנק המרכזי קונה מט''ח ונותן לציבור שקלים $M \uparrow$ (עוקמת $LM$ זזה לחיתוך של $IS = BP$)
            \item חיתוך $IS = LM$ נמצא ימינה / למטה ביחס לעקומת $BP$ - גירעון במאזן התשלומים / עודף ביקוש / $BP < 0$. לכן הבנק המרכזי מוכר מט''ח ולוקח מהציבור שקלים $M \downarrow$ (עוקמת $LM$ זזה לחיתוך של $IS = BP$)

        \end{itemize}
    \end{block}

    \begin{block}{שלב $IV$}
        מחירים משתנים (טווח קצר) עדיין לא חוזרים לתוצר פוטנציאלי. נקודה $D$
        \begin{tikzpicture}

            % Define the positions of the elements
            \node (Y) at (0, 2) {\( Y > Y_P \)};
            \node (P) at (0, 0) {\( P \uparrow \)};
            \node (EP) at (4, 1.5) {\(\frac{EP^*}{P} \downarrow\)};
            \node (M) at (4, 0) {\(\frac{M}{P} \downarrow\)};
            \node (W) at (4, -1.5) {\(\frac{W}{P} \downarrow\)};
            \node (IS) at (8, 1.5) {\( IS \downarrow \,\, BP \downarrow \)};
            \node (LM) at (8, 0) {\( LM \downarrow \)};
            \node (AS) at (8, -1.5) {\( AS \downarrow \)};
            
            % Draw the arrows
            \draw[->, thick, blue] (Y) -- (P);
            \draw[->, thick, blue] (P) -- (EP);
            \draw[->, thick, blue] (P) -- (M);
            \draw[->, thick, blue] (P) -- (W);
            \draw[->, thick, blue] (EP) -- (IS);
            \draw[->, thick, blue] (M) -- (LM);
            \draw[->, thick, blue] (W) -- (AS);
            
            \end{tikzpicture}


           
                
    \end{block}

    \begin{block}{שלב $IV$}
        מחירים משתנים (טווח קצר) עדיין לא חוזרים לתוצר פוטנציאלי. נקודה $D$
        \begin{tikzpicture}

            % Define the positions of the elements
            \node (Y) at (0, 2) {\( Y < Y_P \)};
            \node (P) at (0, 0) {\( P \downarrow \)};
            \node (EP) at (4, 1.5) {\(\frac{EP^*}{P} \uparrow\)};
            \node (M) at (4, 0) {\(\frac{M}{P} \uparrow\)};
            \node (W) at (4, -1.5) {\(\frac{W}{P} \uparrow\)};
            \node (IS) at (8, 1.5) {\( IS \uparrow \,\, BP \uparrow \)};
            \node (LM) at (8, 0) {\( LM \uparrow \)};
            \node (AS) at (8, -1.5) {\( AS \uparrow \)};
            
            % Draw the arrows
            \draw[->, thick, blue] (Y) -- (P);
            \draw[->, thick, blue] (P) -- (EP);
            \draw[->, thick, blue] (P) -- (M);
            \draw[->, thick, blue] (P) -- (W);
            \draw[->, thick, blue] (EP) -- (IS);
            \draw[->, thick, blue] (M) -- (LM);
            \draw[->, thick, blue] (W) -- (AS);
        \end{tikzpicture}

    \end{block}

    \begin{block}{שלב $V$}
        מחירים ושכר משתנים בטווח ארוך, חוזרים לתוצר פוטנציאלי, נקודה $E$

        \begin{tikzpicture}
            
            % Left part
            \node (Y1) at (0, 2) {\( Y > Y_P \)};
            \node (P1) at (0, 0) {\( W \uparrow \,\, P \uparrow \)};
            \node (EP1) at (4, 1.5) {\(\frac{EP^*}{P} \downarrow\)};
            \node (M1) at (4, 0) {\(\frac{M}{P} \downarrow\)};
            \node (W1) at (4, -1.5) {\(\frac{W}{P} \downarrow\)};
            \node (IS1) at (8, 1.5) {\( IS \downarrow \,\, BP \downarrow \)};
            \node (LM1) at (8, 0) {\( LM \downarrow \)};
            \node (AS1) at (8, -1.5) {\( AS \downarrow \)};
            
           
            
            % Draw the arrows for left part
            \draw[->, thick, blue] (Y1) -- (P1);
            \draw[->, thick, blue] (P1) -- (EP1);
            \draw[->, thick, blue] (P1) -- (M1);
            \draw[->, thick, blue] (P1) -- (W1);
            \draw[->, thick, blue] (EP1) -- (IS1);
            \draw[->, thick, blue] (M1) -- (LM1);
            \draw[->, thick, blue] (W1) -- (AS1);
            
           
            
            \end{tikzpicture}
    \end{block}

    \begin{block}{שלב $V$}
        מחירים ושכר משתנים בטווח ארוך, חוזרים לתוצר פוטנציאלי, נקודה $E$

        \begin{tikzpicture}
         % Right part
         \node (Y2) at (14, 2) {\( Y < Y_P \)};
         \node (P2) at (14, 0) {\( W \downarrow \,\, P \downarrow \)};
         \node (EP2) at (18, 1.5) {\(\frac{EP^*}{P} \uparrow\)};
         \node (M2) at (18, 0) {\(\frac{M}{P} \uparrow\)};
         \node (W2) at (18, -1.5) {\(\frac{W}{P} \uparrow\)};
         \node (IS2) at (22, 1.5) {\( IS \uparrow \,\, BP \uparrow \)};
         \node (LM2) at (22, 0) {\( LM \uparrow \)};
         \node (AS2) at (22, -1.5) {\( AS \uparrow \)};


          % Draw the arrows for right part
          \draw[->, thick, blue] (Y2) -- (P2);
          \draw[->, thick, blue] (P2) -- (EP2);
          \draw[->, thick, blue] (P2) -- (M2);
          \draw[->, thick, blue] (P2) -- (W2);
          \draw[->, thick, blue] (EP2) -- (IS2);
          \draw[->, thick, blue] (M2) -- (LM2);
          \draw[->, thick, blue] (W2) -- (AS2);
        \end{tikzpicture}

    \end{block}
    

    \begin{figure}[scale=1.5]
        \centering
        \includegraphics[width=\textwidth]{SCR-20240703-kthq.png}
    \end{figure}
    
    
\end{frame}

\section{סדר פעולות לפתרון בשע''ח נייד}
\begin{frame}[allowframebreaks]
    \frametitle{סדר פעולות לפתרון בשע''ח נייד}
    \begin{block}{שלב $I$}
        סרטוט של מצב המוצא תוך דגש על התייחסות לתנועות הון
        \begin{itemize}
            \item משק סגור לתנועות הון - $ISBP$ עם שיפוע של $IS$ במשק סגור
            \item משק פתוח לתנועות הון מושלמות - $ISBP$ גמישה לחלוטין
            \item משק פתוח לתנועות הון $ISBP$ גמישה יותר מעקומת $IS$
        \end{itemize}
    \end{block}

    \begin{block}{שלב $II$}
        סרטוט של השינוי שחל במשק (טווח מיידי)
        \begin{itemize}
            \item $ISBP$ - $C_0 \uparrow, cT_0 \downarrow , I_0 \uparrow, G_0 \uparrow, TR_0 \downarrow, CF_0 \downarrow, i^{\ *} \uparrow$ \\ הרחבה ימינה ולמעלה
            \item $LM$ - $M_0 \downarrow, M^s \uparrow$ \\ הרחבה ימינה ולמטה
        \end{itemize}
    \end{block}

    \begin{block}{שלב $III$}
        מחירים משתנים, אך עדיין לא חוזרים לתוצר פוטנציאלי. \\
        \begin{tikzpicture}
            % First part: Y > Yp
            \node at (0, 2) {$Y > Y_P$};
            \node at (-1, 0) {$P \uparrow$};
            \draw[->] (-0.5, 0) -- (0.5, 0);
            \node at (1, 0) {$\frac{M}{P} \downarrow$};
            \draw[->] (1.5, 0) -- (2.5, 0);
            \node at (3, 0) {$LM \downarrow$};
        
            % Second part: Y < Yp
            \node at (6, 2) {$Y < Y_P$};
            \node at (5, 0) {$P \downarrow$};
            \draw[->] (5.5, 0) -- (6.5, 0);
            \node at (7, 0) {$\frac{M}{P} \uparrow$};
            \draw[->] (7.5, 0) -- (8.5, 0);
            \node at (9, 0) {$LM \uparrow$};
        \end{tikzpicture}
    \end{block}
    
    \begin{block}{שלב $IV$}
        טווח ארוך, מחירים ושכר משתנים וחוזרים לתוצר פוטנציאלי
        \begin{tikzpicture}
            % First part: Y > Yp
            \node at (0, 2) {$Y > Y_P$};
            \node at (-1, 0) {$P \uparrow, W \uparrow$};
            \draw[->] (-0, 0) -- (0.5, 0);
            \node at (1, 0) {$\frac{M}{P} \downarrow$};
            \draw[->] (1.5, 0) -- (2, 0);
            \node at (2.5, 0) {$LM \downarrow$};
        
            % Second part: Y < Yp
            \node at (6, 2) {$Y < Y_P$};
            \node at (5, 0) {$P \downarrow, W \downarrow$};
            \draw[->] (6, 0) -- (6.5, 0);
            \node at (7, 0) {$\frac{M}{P} \uparrow$};
            \draw[->] (7.5, 0) -- (8, 0);
            \node at (8.5, 0) {$LM \uparrow$};
        \end{tikzpicture}
    \end{block}

\end{frame}
\end{RTL}
\end{document}