% !TEX program = xelatex
\documentclass[10pt,usenames,dvipsnames]{beamer}
\usefonttheme{serif}
\usefonttheme{structuresmallcapsserif}
\usetheme{Warsaw}
\setbeamertemplate{headline}{} % This line removes the headline template
\usepackage{xcolor}
\beamertemplatenavigationsymbolsempty
\usepackage{tikz}
\usepackage{pgfplots}
\renewcommand{\qed}{\hfill\blacksquare}
\newcommand{\D}[1]{\Delta #1}
\renewcommand{\a}{\alpha}
\usepackage{float}
\setbeamertemplate{frametitle continuation}{%
    \ifnum\insertcontinuationcount>999999999 % this command tells the program when to start counting and also the count will be in numbers and not in roman letters
    \insertcontinuationcount
    \fi}
% ============================================================ %
% HEBREW support via polyglossia %
% ============================================================ %
\usepackage{polyglossia}
\defaultfontfeatures{Mapping=tex-text, Scale=MatchLowercase}
\setdefaultlanguage{hebrew}
\setotherlanguage{english}
\newfontfamily\hebrewfont[Script=Hebrew]{David CLM}
\newcommand{\heart}{\ensuremath\heartsuit}
% Use \begin{hebrew} block of text \end{hebrew} for paragraphs.
% Use \texthebrew{ } and \textenglish{ } for short texts.
% ============================================================ %
\title[מאזן התשלומים]{תרגול 7 - מאזן התשלומים}
\author{מתן לבינטוב}
\institute[{{אב"ג}}]{{ אוניברסיטת בן גוריון בנגב}}
\date{}
\usepackage{bidi}
\begin{document}
\begin{RTL}
\begin{frame}
\titlepage
\end{frame}
\begin{frame}
    \frametitle{נושאים}
    \tableofcontents
\end{frame}

\section{מאזן התשלומים}
\begin{frame}[allowframebreaks]
    \frametitle{מאזן התשלומים}
    \begin{block}{סימונים}
    \begin{columns}
        \begin{column}{0.48\textwidth}
            \begin{itemize}
                \item $E$ - שער חליפין נומינלי
                \item $P$ - רמת מחירים מקומית
                \item $P^{\, \ast}$ - רמת מחירים בחו"ל
                \item $\frac{E P^{\, \ast}}{P}$ - שער חליפין ריאלי
                \end{itemize}
        \end{column}
        \begin{column}{0.48\textwidth}
            \begin{itemize}
                \item $i$ - ריבית נומינלית מקומית
                \item $i^{\, \ast}$ - ריבית נומינלית בחו"ל
                \item $X$ - ייצוא
                \item $IM$ - ייבוא
                \item $CF$ - תנועות הון לתוך הארץ
                \end{itemize}
        \end{column}
    \end{columns}
    \end{block}

    \framebreak

    \begin{block}{מאזן התשלומים}
        מאזן התשלומים הוא מסמך אשר מסכם את הכניסה והיציאה של מט''ח מהמשק.

        בשונה ממאזן רגיל, במקום לדון על משתנים מצטברים (מלאי) 
        הוא מתאר משתני זרם.

        מאזן התשלומים בנוי משני חשבונות עיקריים:
        \begin{itemize}
            \item חשבון שוטף - $CA$
            \item חשבון ההון - $CF$
        \end{itemize}
    \end{block}

\end{frame}

\section{חשבון שוטף}
\begin{frame}[allowframebreaks]
    \frametitle{חשבון שוטף}
\begin{block}{חשבון שוטף}
        מכיל בתוכו  את המאזן המסחרי $NX$ ואת העברות החד צדדיות $TR$.
        $$
        CA = NX + TR
        $$
\end{block}

\begin{block}{מאזן מסחרי}
    עודף הייצוא $NX = X - IM$ מסכם את ההכנסות מהייצוא וההוצאות לייבוא.
    עודף הייצוא תלוי בתוצר ובשע''ח ריאלי באופן הבא:
    \begin{align*}
        X &= X_0 + X_1 \cdot \frac{E P^{\, \ast}}{P} \\
        IM &= IM_0 - IM_1 \cdot \frac{E P^{\, \ast}}{P} + IM_2 \cdot Y \\
        NX &= (X_0 - IM_0) + (X_1 - IM_1) \cdot \frac{E P^{\, \ast}}{P} - IM_2  \cdot Y \\
           &= NX_0 + NX_1 \cdot \frac{E P^{\, \ast}}{P} - IM_2 \cdot Y
    \end{align*}
\end{block}

\framebreak

\begin{block}{העברות חד צדדיות}
    העברות של מט''ח שאין נובעות מעסקאות או מניעים כלכליים, לדוגמה למוסדות ללא כוונות רווח
    ולכן הוא גודל אקסוגני.
    $$
    TR = TR_0
    $$
\end{block}
\end{frame}


\section{תנועות הון}
\begin{frame}[allowframebreaks]
    \frametitle{תנועות הון}
    העברות מט''ח אשר נובעות משיקולים פיננסים. השקעה במניות
    השקעה באגרות חוב ואשראי.
    
    תנועות הון תלויות בפער בין הריבית הנומינלית המקומית לריבית הנומילית העולמית, עם התחשבות בציפיות לפיחות.

    \begin{block}{משוואת תנועות הון}
        $$
        CF = CF_0 + CF_1 \cdot \left(i - \left(i^{\, \ast} + \widehat{E}\right)\right)
        $$
        כאשר הריבית המקומית גבוהה מהריבית העולמית, השקעות במשק הן אטרקטיביות יותר ולכן 
        יש תנועות הון לתוך המשק.
        
        התהנליך עובד גם הפוך, כאשר הריבית בחו''ל גבוהה יותר, אז כדאי יותר להשקיע בחו''ל ולכן יש תנועות הון מחוץ למשק.

    \end{block}
    \framebreak
    \begin{alertblock}{ציפיות לפיחות \heart}
        ציפיות לפיחות מורידות את הכדאיות להשקעה בארץ
        משום שפיחות מייקר את המעבר בין שקלים לדולרים ולכן מוריד את התשואה של משקיעים זרים.
    \end{alertblock}
    
    \begin{exampleblock}{דוגמה מספרית}
    \begin{itemize}
        \item נניח משקיע זר מפקיד \(1000\,\$\) בריבית שנתית \(i=5\%\), \(E_0=3\,\$\).
        \item ללא פיחות: \(1000\times1.05\times3=3150\,\mathrm{ILS}\), החזר \(\frac{3150}{3}=1050\,\$\), רווח \(+50\,\$\) (\(+5\%\)).
        \item אם צפויה פיחות \(10\%\): \(E_1=3\times1.1=3.3\,\$\), החזר \(\frac{3150}{3.3}\approx954.55\,\$\), הפסד \(-45.45\,\$\) (\(-4.545\%\)).
        \item מסקנה: פיחות מטבע מקטין משמעותית את ההכנסה ממשקיע זר.
    \end{itemize}
    \end{exampleblock}
    \end{frame}

\section{חוב חיצוני}
\begin{frame}[allowframebreaks]
    \frametitle{חוב חיצוני}
    \begin{block}{השקעות}
        \begin{itemize}
            \item השקעה במכשירי הון - קנייה של מניות או אחוזים בעסק מסוים אשר הופכים את הרוכש לשותף בעסק.
            מה שמשותף במכשירים אלו היא שבדרך כלל לא קיימת התחייבות להחזר של הקרן או תשלומי ריבית מוגדרים מראש.
            \item השקעה במכשירי חוב - קנייה של מכשירי חוב כגון אגרות חוב או מתן הלוואה אשר מקנים לרוכש זכות לקבל בחזרה את הקרן השקעה שלו ובנוסף תשלומי ריבית.
            \end{itemize}
    \end{block}
    \begin{alertblock}{שימו \heart}
        כאשר נמדוד את החוב החיצוני נתעלם ממכשירי הון ונתייחס למכשירי חוב בלבד.
    \end{alertblock}

    \framebreak

    \begin{block}{חוב חיצוני ברוטו}
        כל התחייבויות של המשק המקומי למשקים חיצוניים (מלאי).\\

        שינוי בחוב חיצוני ברוטו הינו $CF$ (משתנה זרם), למה?
        
    \end{block}

    \begin{block}{חוב חיצוני נטו}
        כל התחייבויות של המשק המקומי למשקים חיצונים בניכוי כל התחייבות של המשקים החיצוננים למשק המקומי (נכס מנקודת המבט של המשק המקומי) (משתנה מלאי).


        שינוי בחוב החיצוני נטו $-CA$ (משתנה זרם), למה?
    \end{block}

    נשים לב שהפער בין השינוי בין החוב החיצוני ברוטו לשינוי בחוב החיצוני נטו הוא בדיוק השינוי ביתרות המט''ח.
    $$
    BP = CF + CA
    $$
\end{frame}


\end{RTL}
\end{document}