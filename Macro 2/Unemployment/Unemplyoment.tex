% !TEX program = xelatex
\documentclass{beamer}
\usefonttheme{serif}
\usefonttheme{structuresmallcapsserif}
\usepackage{fontspec}
\usetheme{Warsaw}
\setbeamertemplate{headline}{} % This line removes the headline template
\usepackage{xcolor}
\beamertemplatenavigationsymbolsempty
\usepackage{tikz}
\usepackage{amsmath}
\usetikzlibrary{positioning, arrows.meta, shapes}
\usepackage{pgfplots}
\renewcommand{\qed}{\hfill\blacksquare}
\newcommand{\D}[1]{\Delta #1}
\renewcommand{\a}{\alpha}
\usepackage{float}
\setbeamertemplate{frametitle continuation}{%
    \ifnum\insertcontinuationcount>999999999 % this command tells the program when to start counting and also the count will be in numbers and not in roman letters
    \insertcontinuationcount
    \fi}
% ============================================================ %
% HEBREW support via polyglossia %
% ============================================================ %
\usepackage{polyglossia}
\defaultfontfeatures{Mapping=tex-text, Scale=MatchLowercase}
\setdefaultlanguage{hebrew}
\setotherlanguage{english}
% {Hadasim CLM}
% \newfontfamily{\hebrewfontsf}[Script=Hebrew]{Miriam CLM}
% \newfontfamily{\hebrewfonttt}[Script=Hebrew]{Miriam Mono CLM}

\newfontfamily\hebrewfont[Script=Hebrew]{Arial}
% Use \begin{hebrew} block of text \end{hebrew} for paragraphs.
% Use \texthebrew{ } and \textenglish{ } for short texts.
% ============================================================ %
\title[]{אבטלה בטווח ארוך}
\author{מתן לבינטוב}
\institute[{{ אב"ג}}]{{ אוניברסיטת בן גוריון בנגב}}
\date{}
\usepackage{bidi}
\begin{document}
\begin{RTL}
\begin{frame}
\titlepage
\end{frame}
\begin{frame}
    \tableofcontents[sectionstyle=show,
    subsectionstyle=show/shaded/hide,
    subsubsectionstyle=show/shaded/hide]
\end{frame}

\section{הגדרות}
\begin{frame}[allowframebreaks]{הגדרות}
    \begin{enumerate}
        \item אוכלוסיה - כל האוכלוסייה כולל תינוקות.
        \item אוכלוסיה בגיל העבודה $(N)$ - בני 15+ 
        \item כוח העבודה $(L)$ - כל מי שעובד כרגע $(E)$ וכל מי שלא עובד אבל יכול ומוכן לעבוד ($U$), לא כולל סטודנטים, חיילים בסדיר (החל מ2012 כולל חיילים בקבע), תלמידים , פנסיורים, לא כולל אנשים שחיפשו עבודה והתייאשו.
        \item שיעור השתתפות בכוח העבודה - כוח העבודה / אוכלוסייה בגיל העבודה - $$\frac{U + E}{N} = \frac{L}{N}$$
        \item מובטלים - כל מי ששייך לכוח עבודה אך לא מצא עבודה $\left(U\right)$
        \item שיעור האבטלה - $$\frac{U}{U+E}$$
    \end{enumerate}
    
    \framebreak

    \begin{block}{שימו לב, לאדם מובטל יש 2 אופציות לצאת מהגדרת המובטל}
        \begin{enumerate}
            \item למצוא עבודה ולחזור למעגל העבודה (גורם לירידה במספר המובטלים)
            \item להתייאש מחיפוש עבודה (גורם לירידה גם במספר המובטלים וגם לכוח העבודה)
        \end{enumerate}
    \end{block}
\end{frame}
\section{אבטלה חיכוכית}
\begin{frame}[allowframebreaks]{אבטלה חיכוכית}
    הדינמיקה של שוק העבודה : 
    \begin{center}
        \begin{tikzpicture}[
            box/.style={rectangle, draw, minimum width=2.5cm, minimum height=1cm, fill=orange!30},
            myarrow/.style={-Stealth, thick, black}
        ][H]
        
        \node[box] (box1) {מועסקים};
        \node[box, right=of box1] (box2) {מובטלים};
        
        \draw[myarrow] (box1.north) to[bend left] node[above] {$s \times E$} (box2.north);
        \draw[myarrow] (box2.south) to[bend left] node[below] {$f \times U$} (box1.south);
        
        \end{tikzpicture}    
    \end{center}
    $s$ - הסיכוי של בן אדם מועסק להיות מפוטר \\
    $f$ - הסיכוי של בן אדם מובטל למצוא עבודה

    \framebreak

    \begin{block}{שיעור אבטלה טבעי}
        \begin{equation}
            \Delta U =U_{t + 1} - U_t \ = \underbrace{- f\cdot U_t}_{\text{מובטלים שמצאו עבודה}} + \underbrace{s\cdot E_t}_{\text{מועסקים שפוטרו}}
        \end{equation}

        אנו רוצים למצוא מתי ,$\Delta U = 0$ כלומר, מתי האבטלה היא יציבה, שזה אומר.

            
        \begin{equation}
            - f\cdot U_t + s\cdot E_t = 0 \iff \frac{U}{L}=u^* = \frac{s}{f + s} 
        \end{equation}
        
        מקבלים מ (2) כמה תוצאות די אינטואיטיביות:
        \begin{enumerate}
            \item ככל שהסיכוי לאבד את העבודה הוא גדול יותר, כך שיעורי האבטלה יהיה גדול יותר
            \item ככל שהסיכוי של מובטל למצוא עבודה יהיה גדול יותר, כך שיעורי האבטלה יהיה נמוך יותר
        \end{enumerate}
    \end{block}

    \framebreak

    \begin{block}{סיבות לאבטלה חיכוכית}
        \begin{enumerate}
            \item עובדים ומשרות הטרוגניות
            \item אינפומרציה א-סימטרית
            \item קליטת עובדים כרוכה בהשקעה של זמן וכסף
            \item בררננות מצד מעסיקים ועובדים
        \end{enumerate}
    \end{block}
\end{frame}

\section{אבטלה מבנית}
\begin{frame}[allowframebreaks]
    \frametitle{אבטלה מבנית (כשלים בשוק העבודה)}
    סיבות לאבטלה מבנית : 
    \begin{itemize}
        \item ארגוני עובדים (היצע עבודה לא תחרותי)
        \item מונופסון - מעסיק יחיד (ביקוש לעבודה לא תחרותי)
        \item שכר מינימום
        \item תאוריית שכר הסף
    \end{itemize}

    \framebreak
    \begin{block}{מודל מונופסון}
        המודל דומה למונופול בשוק המוצרים רק הפוך, מטרת המונופסון היא להעסיק את העובדים בצורה הזולה ביות , המעסיק ישווה את הביקוש שלו לעובדים לעלות השולית של העסקת העובדים \\
        שלבים לפיתרון :
        \begin{enumerate}
            \item מציאת עלות כוללת של עובדים (כמות מועסקים כפול מחיר / שכר)
            \item גזירת העלות השולית של העובדים
            \item השוואה בין עלות שולית לביקוש $\impliedby$ זה קובע את כמות העובדים
            \item הצבה של כמות העובדים בגרף היצע $\impliedby$ זה קובע את השכר שישולם
        \end{enumerate}
    \end{block}

    \framebreak

    \begin{block}{תאוריית שכר הסף}
        נתאר פרט עם העדפות תחת כל הנחות שאנחנו רגילים אליהם (תועלת שולית פוחתת , עקומות שמתנהגות יפה) אשר מפיק תועלת משני מוצרים
        \begin{enumerate}
            \item פנאי - $l$
            \item צריכה של מוצרים - $c$
        \end{enumerate}
        מגבלת התקציב היא שעלות צריכת המוצרים שלו צריכה להיות קטנה או שווה לשכר שלו. \\
        בגלל שהעדפות הם מונוטוניות אז אנחנו יודעים בוודאות שהוא יצרוך את כל השכר שלו. \\
        הפרט נהנה אך ורק מפנאי ותצרוכת לכן :
        \begin{align*}
        &\quad \max U(c,l) \\
        & s.t \\
        &\quad c = W \cdot (1 - l)
      \end{align*}
    \end{block}

    שימו לב שאנחנו מנרמלים את יחידת הזמן של הפרט ל1.

    \framebreak

    \begin{exampleblock}{שלבים לפיתרון}
        \begin{enumerate}
            \item מציאת $MRS$ והשוואות לשיפוע קו התקציב (כלומר השכר $W$)
            \item מציאת הביקוש לתנאי ותצרוכת
            \item הצבה ומציאת הביקוש לעבודה
        \end{enumerate}
    \end{exampleblock}

    

\end{frame}
\end{RTL}
\end{document}