% !TEX program = xelatex
\documentclass[usenames,dvipsnames]{beamer}
\usefonttheme{serif}
\usefonttheme{structuresmallcapsserif}
\usetheme{Warsaw}
\setbeamertemplate{headline}{} % This line removes the headline template
\usepackage{xcolor}
\beamertemplatenavigationsymbolsempty
\usepackage{tikz}
\usepackage{pgfplots}
\renewcommand{\qed}{\hfill\blacksquare}
\newcommand{\D}[1]{\Delta #1}
\renewcommand{\a}{\alpha}
\usepackage{float}
\setbeamertemplate{frametitle continuation}{%
    \ifnum\insertcontinuationcount>999999999 % this command tells the program when to start counting and also the count will be in numbers and not in roman letters
    \insertcontinuationcount
    \fi}
% ============================================================ %
% HEBREW support via polyglossia %
% ============================================================ %
\usepackage{polyglossia}
\defaultfontfeatures{Mapping=tex-text, Scale=MatchLowercase}
\setdefaultlanguage{hebrew}
\setotherlanguage{english}
\newfontfamily\hebrewfont[Script=Hebrew]{Arial}
% Use \begin{hebrew} block of text \end{hebrew} for paragraphs.
% Use \texthebrew{ } and \textenglish{ } for short texts.
% ============================================================ %
\title[]{אינפלציה}
\author{מתן לבינטוב}
\institute[{{ אב"ג}}]{{ אוניברסיטת בן גוריון בנגב}}
% \date{}
\usepackage{bidi}
\begin{document}
\begin{RTL}
\begin{frame}
\titlepage
\end{frame}
\begin{frame}
    \frametitle{נושאים}
    \tableofcontents

    

\end{frame}

\section{דגשים}
\begin{frame}
    \frametitle{דגשים}
    \begin{block}{דגשים לגבי אינפלציה}
        \begin{itemize}
            \item הדפסת כסף חד פעמית מביאה לעליית מחירים חד פעמית ולא לאינפלציה
            \item הדפסת כסף בשיעור קבוע מביאה לאינפלציה
            \item הדפסת כסף לא גורמת לשינויים ריאלים בטווח הארוך אך כן לשינויים בטווח הקצר
        \end{itemize}
    \end{block}
    

\end{frame}

\section{גרפים}
\begin{frame}[allowframebreaks]
    \frametitle{הרחבה מוניטרית חד פעמית}
    \begin{center}
        \begin{tikzpicture}[scale=1.3, thick]
            % Axes
            \draw [-] (0,0) -- (5,0) node [right] {$Y$};
            \draw [-] (0,0) -- (0,5) node [above] {$i$};
            
            % IS curve
            \draw [blue] (1,4) -- (4,1) node [below right] {$IS$};
            
            % LM curve
            \draw [blue] (1,1) -- (4,4) node [above right] {$LM$};
            \draw [red] (1,0.25) -- (5,4.25) node [above right] {$LM^{\ \prime}$};


            % Draw line at Y_bar
            \draw [black] (2.5,0) -- (2.5,5);
            % Labels
            % \node [below] at (0,0) {$0$};
            \node [below] at (2.5,0) {$\bar{Y}$};
            
            % Dotted lines
            \draw [dotted] (2.5,0) -- (2.5,2.5);
            \draw [dotted] (0,2.5) -- (2.5,2.5);
        
          \end{tikzpicture}
    \end{center}

    \framebreak

    \begin{block}{מה יקרה?}
        \begin{enumerate}
            \item $LM$ זזה ימינה
            \item בטווח המיידי הכל קשיח ולכן שום דבר לא משתנה
            \item בטווח הקצר המחירים נהפכים לגמישים ולכן תהיה עליית מחירים שתביא לקיזוז חלקי של LM
            \item בטווח הארוך גם השכר נהפך לגמיש ולכן ההרחבה המוניטרית מתבטלת וחוזרים לאותה רמה של ריבית ותוצר
        \end{enumerate}
    \end{block}
    
    

\end{frame}


\begin{frame}[allowframebreaks] % FIX
    \frametitle{הרחבה פיסקלית חד פעמית}
    \begin{center}
        \begin{tikzpicture}[scale=1.3, thick, dot/.style={circle,inner sep=1pt,fill,name=#1}]

            % Axes
            \draw [-] (0,0) -- (5,0) node [right] {$Y$};
            \draw [-] (0,0) -- (0,5) node [above] {$i$};
            
            % IS curves
            \draw [blue] (0.5,3.5) -- (3,1) node [below right] {$IS(G_1)$};
            \draw [red] (1.5,3.5) -- (4,1) node [below right] {$IS(G_2)$};
            
            % LM curves
            \draw [blue] (0.5,1) -- (3.5,4) node [above right] {$LM(\frac{M}{P_1})$};
            \draw [red] (-0.5,1) -- (2.5,4) node [above left] {$LM(\frac{M}{P_2})$};
          
            % Equilibrium points
            \node[dot=A] at (1.75,2.25) {};
            \node[dot=B] at (2.25,2.75) {};
          
            % Dotted lines
            \draw [dotted] (A) -- (1.75,0) node [below] {$Y_1$};
            \draw [dotted] (B) -- (2.25,0) node [below] {$Y_2$};
            \draw [dotted] (A) -- (0,2.25) node [left] {$i_1$};
            \draw [dotted] (B) -- (0,2.75) node [left] {$i_2$};
            
            % Labels
            % \node [below] at (0,0) {0};
          
          \end{tikzpicture}
    \end{center}
    \framebreak
    \begin{block}{מה יקרה?}
        \begin{enumerate}
            \item $IS$ זזה ימינה
            \item בטווח המידיי הכל קשיח ולכן שום דבר לא משתנה
            \item בטווח הארוך גם השכר וגם המחירים נהפכים לגמישים, בגלל עודף הביקוש יש עליית מחירים מה שגורם להזזה שמאלה של עקומת $LM$ ולכן חוזרים לאותה רמת תוצר אבל עם ריבית גבוהה יותר
        \end{enumerate}
    \end{block}

\end{frame}

\section{תורת הכמות של הכסף}
\begin{frame}[allowframebreaks]
    \frametitle{תורת הכמות של הכסף}
    \begin{block}{תורת הכמות של הכסף}
        $$M \times V = P \times Y \implies \underbrace{\widehat{M}}_{m} + \widehat{V} = \underbrace{\widehat{P}}_{\pi} + \widehat{Y} \implies  m = \widehat{P} + \widehat{Y} - \widehat{V}$$
    \end{block}
    
    \begin{block}{הגישה הקלאסית}
        הגישה הקלאסית אומרת אומרת שבטווח הקצר והארוך $\widehat{V} = 0$ ולפי תורת כמות הכסף נקבל :
        $$\frac{M^d}{P} = \frac{Y}{V} = KY$$
    \end{block}

    \begin{block}{גישה הקייסיאנית (העדפת הנזילות)}
        $$\frac{M^d}{P} = M_0 + KY - hi$$
        ובטווח הקצר $\widehat{V} \neq 0$ אך בטווח הארוך $\widehat{V} = 0$
    \end{block}

\end{frame}


\section{אינפלציה בטווח הארוך}

\section{גורמים לאינפלציה}
\begin{frame}[allowframebreaks]
    \frametitle{גורמים לאינפלציה}
    \begin{itemize}
        \item הדפסת כסף $\impliedby$ יצירת ציפיות לאינפלציה $\impliedby$ אינפלציה
        \item גידול בהוצאות הממשלתיות $\impliedby$ עליית ריבית $\impliedby$ הדפסת כסף $\impliedby$ יצירת ציפיות לאינפלציה $\impliedby$ אינפלציה
        \item גריעון $\impliedby$ עלייה בחוב $ \impliedby$ הדפסת כסף למימון החוב $\impliedby$ יצירת ציפיות לאינפלציה $\impliedby$ אינפלציה
    \end{itemize}
    
    \framebreak

    \begin{block}{תזכורת לגבי אג''ח}
        \begin{itemize}
        \item  אג''ח צמוד - משלם ריבית ריאלית + הצמדה למדד המחירים $\impliedby$ כלומר הוא לא מושפע מהפערים בין אינפלציה צפויה לאינפלציה בפועל
        \item אג''ח לא צמוד - משלם ריבית נומינלית $\impliedby$ כלומר הוא כן מושפע מפערים בין אינפלציה צפויה לאינפלציה בפועל
        \end{itemize}
        
    \end{block}

\end{frame}
\end{RTL}
\end{document}