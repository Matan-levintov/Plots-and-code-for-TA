% !TEX program = xelatex
\documentclass[usenames,dvipsnames]{beamer}
\usefonttheme{serif}
\usefonttheme{structuresmallcapsserif}
\usetheme{Warsaw}
\setbeamertemplate{headline}{} % This line removes the headline template
\usepackage{xcolor}
\beamertemplatenavigationsymbolsempty
\usepackage{tikz}
\usepackage{pgfplots}
\renewcommand{\qed}{\hfill\blacksquare}
\newcommand{\D}[1]{\Delta #1}
\renewcommand{\a}{\alpha}
\usepackage{float}
\usepackage{graphicx}
\setbeamertemplate{frametitle continuation}{%
    \ifnum\insertcontinuationcount>999999999 % this command tells the program when to start counting and also the count will be in numbers and not in roman letters
    \insertcontinuationcount
    \fi}
\newcommand{\heart}{\ensuremath\heartsuit}
% ============================================================ %
% HEBREW support via polyglossia %
% ============================================================ %
\usepackage{polyglossia}
\defaultfontfeatures{Mapping=tex-text, Scale=MatchLowercase}
\setdefaultlanguage{hebrew}
\setotherlanguage{english}
\newfontfamily\hebrewfont[Script=Hebrew]{Arial}
% Use \begin{hebrew} block of text \end{hebrew} for paragraphs.
% Use \texthebrew{ } and \textenglish{ } for short texts.
% ============================================================ %
\title[]{משק פתוח עם שע''ח קבוע באבטלה}
\author{מתן לבינטוב}
\institute[{{ אב"ג}}]{{ אוניברסיטת בן גוריון בנגב}}
\date{}
\usepackage{bidi}
\begin{document}
\begin{RTL}
\begin{frame}
\titlepage
\end{frame}
\begin{frame}
    \frametitle{נושאים}
    \tableofcontents
    

\end{frame}


\section{עקומות}

\begin{frame}[allowframebreaks]
    \frametitle{עקומות}
    \begin{block}{עקומת $IS$}
        $$Y = \frac{1}{1-  (c - im)} \left(A - b\cdot i + n \cdot e\right)$$
    \end{block}
    \begin{alertblock}{שימו \heart}
        המכפיל של משק פתוח הוא קטן יותר מהמכפיל של משק סגור כתוצאה מקיזוז של יבוא
    \end{alertblock}
    \begin{block}{עקומת $LM$}
        $$i = \frac{1}{h}\left(kY + M_0 - \frac{M}{P}^d\right)$$
    \end{block}
    \begin{block}{עקומת $BP$}
        $$Y = \frac{NX_0 + n\cdot E + TR_0 + CF_0 + cf\left(i-i^* - \widehat{E}^e\right)}{im}$$
    \end{block}

    \framebreak
    \begin{block}{השיפוע של עקומת $BP$}
        $$\frac{dY}{di} = \frac{cf}{im}$$
    \end{block}

    \framebreak
    \begin{center}
        \begin{tikzpicture}[scale=1.2]
            % Define axes
            \draw[thick,->] (0,0) -- (6,0) node[anchor=north west] {$Y$};
            \draw[thick,->] (0,0) -- (0,4) node[anchor=south east] {$i$};
        
            % Define lines
            \draw (0.5,1) -- (5.5,3) node[anchor=west] {$BP = 0$ \text{תשלומים במאזן איזון}};
        
            % Annotations
            \node at (2,2.5) {$BP>0$ \text{תשלומים במאזן עודף}};
            \node at (4.5,1) {$BP<0$ \text{תשלומים במאזן גירעון}};

        
        \end{tikzpicture}
        
    \end{center}
$$BP(NX_0^{\ +},E^{\ +},TR_0^{\ +},CF_0^{\ +},{i^{\ *}}^{\ -})$$
\end{frame}

\section{מצבי קיצון ב $ BP$}
\begin{frame}
    \frametitle{$BP, cf = \infty$}
    \begin{center}
        \begin{tikzpicture}[scale=1.3]
            % Define axes
            \draw[thick,->] (0,0) -- (6,0) node[anchor=north west] {$Y$};
            \draw[thick,->] (0,0) -- (0,4) node[anchor=south east] {$i$};
        
            % Define lines
            \draw (0.5,2) -- (5.5,2) node[anchor=west] {$BP = 0$ \text{תשלומים במאזן איזון}};
        
            % Annotations
            \node at (2,3) {$BP>0$ \text{תשלומים במאזן עודף}};
            \node at (4.5,1) {$BP<0$ \text{תשלומים במאזן גירעון}};

        
        \end{tikzpicture}
        
    \end{center}

    

\end{frame}

\begin{frame}
    \frametitle{$BP, cf = 0$}
    \begin{center}
        \begin{tikzpicture}[scale=1.5]
            % Define axes
            \draw[thick,->] (0,0) -- (6,0) node[anchor=north west] {$Y$};
            \draw[thick,->] (0,0) -- (0,4) node[anchor=south east] {$i$};
        
            % Define lines
            \draw (3,0) -- (3,5) node[anchor=west] {$BP = 0$ \text{תשלומים במאזן איזון}};
        
            % Annotations
            \node at (1.5,2.5) {$BP>0$  \text{תשלומים במאזן עודף}};
            \node at (5,1) {$BP<0$   \text{תשלומים במאזן גירעון}};

        
        \end{tikzpicture}
        
    \end{center}

    

\end{frame}

\section{איך לפתור?}
\begin{frame}
    \frametitle{איך לפתור?}
    \begin{exampleblock}{סדר פעולות לפיתרון}
        \begin{enumerate}
            \item בניית שלושת העקומות
            \item מציאת ש''מ $IS = BP$
            \item עקומת $LM$ מתאימה את עצמה לנקודת החיתוך בין $IS$ ל $BP$
        \end{enumerate}
    \end{exampleblock}

    \begin{block}{משוואת הקסם}
        $$\D Y = \D C + \D I  + \D G + \D NX$$
        $$0 = \D NX + \D TR + \D CF$$
        $$\D NX = \D NX_0 - im \times \D Y + n \times \D e$$
    \end{block}
    
    \begin{alertblock}{שימו \heart}
        כאשר המשק נוקט במדיניות של שע"ח קבוע לבנק אין שליטה על כמות הכסף אך יש
לו שליטה על שע"ח נומינלי. המדיניות המוניטרית של הבנק המרכזי נעשית באמצעות
פיחות/ייסוף ולא באמצעות הרחבה/צמצום מוניטרי/ת.
    \end{alertblock}

\end{frame}
\end{RTL}
\end{document}