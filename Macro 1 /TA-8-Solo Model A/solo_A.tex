% !TEX program = xelatex
\documentclass[usenames,dvipsnames, 10pt]{beamer}
\usefonttheme{serif}
\usefonttheme{structuresmallcapsserif}
\usetheme{Warsaw}
\setbeamertemplate{headline}{} % This line removes the headline template
\usepackage{xcolor}
\beamertemplatenavigationsymbolsempty
\usepackage{tikz}
\usepackage{pgfplots}
\pgfplotsset{compat=1.17}
\renewcommand{\qed}{\hfill\blacksquare}
\newcommand{\D}[1]{\Delta #1}
\renewcommand{\a}{\alpha}
\usepackage{float}
\usepackage{amsmath}
\usepackage{amssymb}
\usepackage{geometry}
\definecolor{darkgreen}{rgb}{0.0, 0.5, 0.0}
\setbeamertemplate{frametitle continuation}{%
    \ifnum\insertcontinuationcount>999999999 % this command tells the program when to start counting and also the count will be in numbers and not in roman letters
    \insertcontinuationcount
    \fi}
% ============================================================ %
% HEBREW support via polyglossia %
% ============================================================ %
\usepackage{polyglossia}
\defaultfontfeatures{Mapping=tex-text, Scale=MatchLowercase}
\setdefaultlanguage{hebrew}
\setotherlanguage{english}
\newfontfamily\hebrewfont[Script=Hebrew]{David CLM}
% Use \begin{hebrew} block of text \end{hebrew} for paragraphs.
% Use \texthebrew{ } and \textenglish{ } for short texts.
% ============================================================ %
\title{תרגול - צמיחה א' - סולו א'}
\author{\texthebrew{ מתן לבינטוב}}
\institute[{{אב"ג}}]{{ אוניברסיטת בן גוריון בנגב}}
\date{}
\usepackage{bidi}
\begin{document}
\begin{RTL}
\begin{frame}
\titlepage
\end{frame}
\begin{frame}
    \frametitle{נושאים}
    \tableofcontents

    

\end{frame}

\section{מודל סולו}
\begin{frame}[allowframebreaks]
    \frametitle{מודל סולו}
    מודל סולו הוא מודל קלאסי אשר מתאר שיווי משקל דינאמי של צד היצע של כלכלה
    \begin{itemize}
        \item בטווח האורך  ניתן לראות שהתוצר, כמות העובדים, מלאי ההון, הצריכה והחיסכון צומחים בקצב קבוע והגדלים הריאלים נותרים ללא שינו, למצב כזה אנו קוראים Steady State
        \item בטווח הקצר ניתן לראות שהתוצר, כמות העובדים, כמות מלאי ההון, הצריכה והחסכון צומחות
        בקצב משתנה מתקופה לתקופה והגדלים הריאליים תוצר לעובד, מלאי ההון לעובד, צריכה
        לעובד וחיסכון לעובד  משתנים גם הם. למצב הזה נקרא התכנסות לשיווי משקל דינאמי.
    \end{itemize}
    \begin{alertblock}{חשוב}
        בגדלים ריאלים הכוונה ל - תוצר לעובד, מלאי ההון לעובד, צריכה לעובד
        וחיסכון לעובד
    \end{alertblock}

    \framebreak

    % \begin{block}{סימונים מקובלים}
    %     \begin{align*}
    %         & y = x = \frac{Y}{L} & \text{תוצר לעובד} \\ & k = \frac{K}{L} & \text{מלאי הון לעובד} \\ & \frac{D}{L} = d \cdot \frac{K}{L} = dk & \text{פחת לעובד} \\ & \frac{S}{L} = \frac{sY}{L} = s \cdot y & \text{חיסכון לעובד} \\ & \frac{C}{L} = \frac{(1-s) Y}{L} = (1-s) \cdot y & \text{צריכה לעובד} \\ & \hat{L} = \frac{\dot L}{L} = n & \text{שיעור הריבוי} \\ & I_g = S & \text{השקעה גולמית} \\ & \dot K = I_N = I_g - D & \text{השקעה נקייה} \\ & MPK = i_c = r + d  & \text{מחיר ההון } 
    %     \end{align*}
    % \end{block}
    
    \framebreak


    % \begin{align*}
    % \text{התוצר לעובד} & & y &= \frac{Y}{L} \\
    % \text{הון לעובד} & & k &= \frac{K}{L} \\
    % \text{פריון כולל} & & \frac{\dot{L}}{L} &= i = n \\
    % \text{פריון פיזי} & & d &= \frac{D}{K} = dk \\
    % \text{השקעה גולמית} & & I_g &= S \\
    % \text{השקעה נטו} & & \dot{K} &= I_n = I_g - D \\
    % \text{מחיר ההון} & & MP_K &= i_c = \frac{R}{P} = r + d \\
    % \text{מחיר העבודה} & & s &= \frac{sY}{L} = s \cdot y \\
    % \text{עלות לעובד} & & c &= \frac{(1-s)Y}{L} = (1 - s) \cdot y \\
    % \end{align*}

    \framebreak
    
    \begin{block}{סימונים מקובלים}
        \begin{minipage}[t]{0.5\textwidth}
            \begin{align*}
            \hat{L} &\equiv \frac{\dot{L}}{L} = n \quad \text{שיעור הריבוי} \\
            I_g &= S \quad \text{השקעה גולמית} \\
            \dot{K} &= I_n = I_g - D \quad \text{השקעה נקייה}\\
            MPK &= i_c = r + d \quad \text{מחיר ההון}
            \end{align*}
        \end{minipage}%
        \begin{minipage}[t]{0.5\textwidth}
            \begin{align*}
            y &\equiv x \equiv  \frac{Y}{L} \quad \text{תוצר לעובד} \\
            k &\equiv \frac{K}{L} \quad \text{מלאי הון לעובד} \\
             \frac{D}{L} &\equiv dk \quad  \text{פחת לעובד} \\
            \frac{S}{L} &\equiv \frac{sY}{L} \equiv s \cdot y \quad \text{חיסכון לעובד} \\
            \frac{C}{L} &\equiv \frac{(1-s)Y}{L} \equiv (1 - s) \cdot y \quad \text{צריכה לעובד}
            \end{align*}
        \end{minipage}
    \end{block}

\end{frame}


\section{פונקצית היצור}
\begin{frame}
    \frametitle{פונקציית יצור}
    \begin{block}{Cobb-Douglas}
        במודל נבחר פונקציית יצור מסוג קוב - דאגלס אשר מקיימת תק''ל
        \[Y = A K^\alpha L ^{1-\alpha}\]
        אם נעבור לדבר במונחים של תוצר לעובד
        \[y \equiv \frac{Y}{L} = \frac{ A K^\alpha L ^{1-\alpha}}{L} = A \left (\frac{K}{L} \right )^\alpha = A k^\alpha\]
    \end{block}
\end{frame}

\section{משוואת התנועה}

\begin{frame}
    \frametitle{משוואת התנועה}
    \begin{block}{משוואת התנועה}
        \begin{equation*}
            \dot k = sy - (n+d)\cdot k
        \end{equation*}
        מתוך המשוואה הזאת אנו מקבלים 3 אפשריות
        \begin{enumerate}
            \item $\impliedby sy > (n+d) \cdot k  $ $\dot k > 0$ משמע השקעה לעובד גדולה יותר משחיקת ההון לעובד ולכן מלאי ההון לעובד גדל. צבירה חיובית של הון לעובד
            \item $\impliedby sy < (n+d) \cdot k  $ $\dot k < 0$ משמע השקעה לעובד קטנה יותר משחיקת ההון לעובד ולכן מלאי ההון לעובד קטן. צבירה שלילית של הון לעובד
            \item $\impliedby sy = (n+d) \cdot k  $ $\dot k = 0$ משמע השקעה לעובד שווה לשחיקת ההון לעובד ולכן מלאי ההון לעובד לא משתנה.
        \end{enumerate}
    \end{block}
\end{frame}



\section{מצב יציב - S.S - Steady State}
\begin{frame}[allowframebreaks]
    \frametitle{מצב יציב}
    \begin{block}{מצב יציב}
        \begin{equation}
            k_{ss} = \left(\frac{sA}{n+d}\right) ^ {\frac{1}{1-\alpha}}
        \end{equation}
        \begin{equation}
            y_{ss} = A \cdot k_{ss}^\alpha
        \end{equation}
        
    \end{block}
    
    \begin{alertblock}{הערות}
        \begin{enumerate}
            \item המשוואה השנייה היא היא פשוט המשוואה הרגילה אבל עם הצבה של $k_{ss}$
            \item כפי שניתן לראות ש''מ ישתנה רק אם יהיה שינוי ב $A,d,n,s$
            \item מדינות עם נתוני משק זהים התכנסו לאותו ש''מ
        \end{enumerate}
    \end{alertblock}
    \framebreak
    \begin{block}{במצב עמיד ישנה צמיחה מאוזנת}
        \begin{equation*}
            \hat k = 0  \implies \widehat K = \widehat{ \left (k \cdot L \right )} = \hat k + \hat L = \underbrace{\hat k}_{\hat k = 0} + n = n
        \end{equation*}
        \begin{equation*}
            \hat y = 0 \implies \widehat Y = \widehat{ \left (y \cdot L \right )} = \hat y + \hat L = \underbrace{\hat y}_{\hat y = 0} + n = n
        \end{equation*}
        
        
    \end{block}

    \begin{alertblock}{במצב לא עמיד אין צמיחה מאוזנת}
        \begin{equation*}
            \hat k \ne 0  \implies \hat K = n + \hat k \ne n
        \end{equation*}
        \begin{equation*}
            \hat y \ne 0 \implies \hat{Y} = n + \hat y \ne n
        \end{equation*}
    \end{alertblock}
\end{frame}

\section{נגזרות}

\begin{frame}
    \frametitle{נגזרות}
    \begin{alertblock}{הערה לגבי נגזרת}
        בשביל לגזור תוצר לעובד לפי זמן, יש להשתמש בכלל שרשרת. \\
        זאת משום שהתוצר לעובד לא תלוי ישירות בזמן, אלא הוא תלוי בהון לעובד $k$ שתלוי בזמן,
        \begin{equation*}
            \frac{d y}{d t} \equiv \dot y = \frac{d y}{d k } \cdot \frac{d k}{d t} = mpk \cdot \dot k 
        \end{equation*}
        $$\implies \hat y = \frac{mpk \cdot \dot k }{y}$$
    \end{alertblock}
    

\end{frame}



\section{סטיות ויציאה מש''מ}
\begin{frame}
    \frametitle{גרף}
    \begin{center}
        \begin{tikzpicture}
        \begin{axis}[
            % title={סולו מודל},
            xlabel={$k$},
            ylabel={$y$},
            xmin=0, xmax=10,
            ymin=0, ymax=10,
            axis lines=left,
            % grid=major,
        ]

        % Production function Y = F(K)
        \addplot[
            domain=0:10, 
            samples=100, 
            color=red,
        ]{2*x^(0.5)}; % An example production function, Y = K^0.5
        \addlegendentry{$y=f(k)$}

        % Depreciation line
        \addplot[
            domain=0:10, 
            samples=100, 
            color=blue,
        ]{3*x*0.1}; % Depreciation line, example 10% of K
        \addlegendentry{לעובד הון שחיקת = $n+d$}

        % Savings function
        \addplot[
            domain=0:10, 
            samples=100, 
            color=darkgreen,
        ]{2*0.3*x^(0.5)}; % Savings function, example 30% of Y
        \addlegendentry{לעובד חיסכון = $sy = sf(k)$}
        
        \end{axis}
        \end{tikzpicture}
    \end{center}
    

\end{frame}

\begin{frame}
    \frametitle{עוד גרף}
    \begin{figure}[h!]
        \centering % Center the plot in the document
        \begin{tikzpicture}
        \begin{axis}[
            axis lines=left, % Position of axis lines
    ticks=none, % Removes the ticks and tick labels
    clip=false, % Prevents clipping of elements outside the axis
    xmax=10, ymax=3, % Defines the range of the axes
    every axis y label/.style={at={(axis description cs:-0.1,1.05)},anchor=south}, % Positions the y-axis label
    every axis x label/.style={at={(axis description cs:1.05,-0.1)},anchor=west}, % Positions the x-axis label
    xlabel={$k$}, % Set the label for the x-axis
    ylabel={$y$}, % Set the label for the y-axis
        ]
        
        % Add the 'n+d' line
        \addplot [domain=0:10, samples=100, smooth, thick, red] {0.1*x};
        \addlegendentry{$n+d$}
        
        % Add the 'y=k^a' line
        \addplot [domain=0:10, samples=100, smooth, thick, black] {x^0.5};
        \addlegendentry{$y=k^{\alpha}$}
        
        % Add the 'sy=sk^a' line
        \addplot [domain=0:10, samples=100, smooth, thick, darkgreen] {0.2*x^0.5};
        \addlegendentry{$sy=sk^{\alpha}$}
        
        % Add vertical dashed lines
        \draw [dashed,blue] (axis cs:2,0) -- (axis cs:2,2^0.5);
        \draw [dashed] (axis cs:4,0) -- (axis cs:4,4^0.5);
        \draw [dashed,red] (axis cs:6,0) -- (axis cs:6,6^0.5);
        
        % Add horizontal dashed lines
        \draw [dashed,blue] (axis cs:0,2^0.5) -- (axis cs:2,2^0.5);
        \draw [dashed] (axis cs:0,4^0.5) -- (axis cs:4,4^0.5);
        \draw [dashed,red] (axis cs:0,6^0.5) -- (axis cs:6,6^0.5);
        
        % Add annotations for 'Kss' and 'Yss'
        \node at (axis cs:2,-0.5) {$k_{ss}$};
        \node at (axis cs:4,-0.5) {$k_{ss}$};
        \node at (axis cs:6,-0.5) {$k_{ss}$};
        \node at (axis cs:-0.5,2^0.5) {$y_{ss}$};
        \node at (axis cs:-0.5,4^0.5) {$y_{ss}$};
        \node at (axis cs:-0.5,6^0.5) {$y_{ss}$};
        
        \end{axis}
        \end{tikzpicture}
        % \caption{Plot of the economic functions} % Caption for the plot
        \end{figure}
    

\end{frame}

\begin{frame}
    \frametitle{סטיות ויציאה מש''מ}
    \begin{alertblock}{הצבר הון שלילי}
        כאשר אנחנו ב $k$ גבוהה יותר מ $k_{ss}$ ולכן השקעה הנדרשת  היא גבוהה יותר מההשקעה בפועל
        $$
        \hat k < 0 \quad \hat y < 0 \quad \hat Y < n 
        $$
    \end{alertblock}

    \begin{exampleblock}{הצבר הון חיובי}
        כאשר אנחנו ב $k$ שהוא נמוך יותר מ $k_{ss}$ לכן השקעה בפועל היא גדולה יותר מהשקעה הנדרשת
        \begin{equation*}
            \hat k > 0 \quad \hat y > 0 \quad \hat Y > n 
        \end{equation*}
    \end{exampleblock}
    

\end{frame}


\end{RTL}
\end{document}