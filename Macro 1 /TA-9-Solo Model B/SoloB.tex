% !TEX program = xelatex
\documentclass[usenames,dvipsnames]{beamer}
\usefonttheme{serif}
\usefonttheme{structuresmallcapsserif}
\usetheme{Copenhagen}
\setbeamertemplate{headline}{} % This line removes the headline template
\usepackage{xcolor}
\beamertemplatenavigationsymbolsempty
\usepackage{tikz}
\usepackage{pgfplots}
\renewcommand{\qed}{\hfill\blacksquare}
\newcommand{\D}[1]{\Delta #1}
\renewcommand{\a}{\alpha}
\usepackage{float}
\setbeamertemplate{frametitle continuation}{%
    \ifnum\insertcontinuationcount>999999999 % this command tells the program when to start counting and also the count will be in numbers and not in roman letters
    \insertcontinuationcount
    \fi}

% ============================================================ %
% HEBREW support via polyglossia %
% ============================================================ %
\usepackage{polyglossia}
\defaultfontfeatures{Mapping=tex-text, Scale=MatchLowercase}
\setdefaultlanguage{hebrew}
\setotherlanguage{english}
\newfontfamily\hebrewfont[Script=Hebrew]{Arial}
% Use \begin{hebrew} block of text \end{hebrew} for paragraphs.
% Use \texthebrew{ } and \textenglish{ } for short texts.
% ============================================================ %
\title{תרגול 10 - צמיחה ב'}
\author{\texthebrew{ מתן לבינטוב}}
\institute[{{ אב"ג}}]{{ אוניברסיטת בן גוריון בנגב}}
\date{20 לדצמבר 2023}
\usepackage{bidi}
\begin{document}
\begin{RTL}
\begin{frame}
\titlepage
\end{frame}
\begin{frame}
    \frametitle{נושאים}
    \tableofcontents
\end{frame}

\section{אופטימום חברתי}


\begin{frame}[allowframebreaks]
    \frametitle{אופטימום חברתי}
    \begin{block}{הגדרה}
        לפי הגדרה, אופטימום חברתי הינו מצב יציב, כלומר Steady State, שבוא רמת החיים, שבמודל שלנו היא הצריכה לעובד היא מקסימלית. \\
        קיימים 3 תנאים שמתקיימים באופטימום חברתי : 
        \begin{itemize}
            \item $MPK^{GR} = n + d  $, כלומר שיפוע המשיק בנקודה שווה לשיפוע של הקו שחיקת ההון
            \item $s = \alpha$ שיעור החיסכון שווה לחזקה של ההון בפונקציית היצור
            \item כלל הזהב שמתקבל הינו : $r = n$, כלומר הריבית שווה לשיעור הריבוי
        \end{itemize}  
    \end{block}

    \framebreak

    \begin{alertblock}{חיסכון חסר}
        $$s < \alpha \implies k < k^{GR} \implies MPK > MPK^{GR} \implies r > n$$
        פיתרון הינו הגדלת החיסכון על ידי צמצום פיסקלי או העלת מיסים 
        $$G \downarrow  / T \uparrow \implies s \uparrow \implies k \uparrow \implies MPK \downarrow \implies r \downarrow$$
    \end{alertblock}

    \begin{exampleblock}{חיסכון יתר}
        $$s > \alpha \implies k > k^{GR} \implies MPK < MPK^{GR} \implies r < n$$   
        הפיתרון הינו צמצום החיסכון על ידי הורדת מיסים או הרחבה פיסקלית
        $$G \uparrow  / T \downarrow \implies s \downarrow \implies k \downarrow \implies MPK \uparrow \implies r \uparrow$$
    \end{exampleblock}
    

\end{frame}

\section{שיפורים טכנולוגיים}

\begin{frame}[allowframebreaks]
    \frametitle{שיפורים טכנולוגיים}
    \begin{block}{מודל העובד המתייעל}
        קיים מקדם יעילות $E$ שמסמן את יעילות כוח העבודה $L$, נניח שמקדם היעילות צומח בקצב קבוע $g$. \\
        לכן פונקציית היצור החדשה נראת כך : 
        $$Y =  A K^{\alpha} \left(EL\right) ^ {1-\alpha}$$
        לכן נעבור מלדבר על עובדים לעובדים יעילים, כלומר במקום לחלק ב$L$ נחלק ב $EL$.
        $$\tilde{y} = A \tilde{k}^\alpha$$

    \end{block}
    מושגים : 
    \begin{itemize}
        \item תוצר לעובד אפקטיבי \quad $\dfrac{Y}{LE} = \dfrac{y}{E} = \tilde{y}$
        \item הון לעובד אפקטיבי \quad $\dfrac{K}{LE} = \dfrac{k}{E} = \tilde{k}$
    \end{itemize}


    \framebreak

    \begin{block}{שיעור השינוי ב - $SS$}
        \begin{equation*}
            \begin{aligned}
            & \hat{\tilde{y}}=\hat{\tilde{k}}=0 \\
            & \hat{k}=\hat{y}=\widehat{(\tilde{y} E)}=\hat{\tilde{y}}+\hat{E}=0+g=g \\
            & \widehat{K}=\hat{Y}=(\widehat{\tilde{y} L E})=\hat{\tilde{y}}+\hat{L}+\hat{E}=0+n+g=n+g
        \end{aligned}
        \end{equation*}
    \end{block}
    מכך שמשוואת התנועה החדשה היא:
    \begin{equation*}
        \dot{\tilde{k}}=s \tilde{y}-(n+d+g) \tilde{k}
    \end{equation*}
    ו $\tilde{k_{ss}}$ הוא : 
    \begin{equation*}
        \tilde{k}_{s s}=\left[\frac{s A}{n+d+g}\right]^{\frac{1}{1-\alpha}}
    \end{equation*}

    \framebreak

\end{frame}

\section{מסקנות חשובות בפונקציית קוב דאגלס}

\begin{frame}[allowframebreaks]
    \frametitle{מסקנות חשובות בפונקציית קוב דאגלס}
    \begin{itemize}
        \item במצב עמיד הריבית הריאלית קבועה : 
        $$r = mpk - d = A\alpha \tilde{k} ^ {\alpha - 1 } - d $$
        \item לעומת המודל המקורי, במקרה הזה, השכר צומח בקצב $g$ $$\frac{W}{P} = MPL = A E \left(1 - \alpha \right) \tilde{k} ^ \alpha$$
        \item התמורה היחסית של כל גורם יצור קבועה לאורך הזמן (זה נכון בכללי עם קוב דאגלס ולא ספציפית בגלל העובד המתייעל) : $$S_k = \frac{MPK \cdot K}{Y } = \frac{A \alpha K^{\alpha - 1 } L ^{1 - \alpha} K }{A K^\alpha L^{1-\alpha}} = \alpha$$ $$
        {S}_L=\frac{M P L * L}{Y}=\frac{A(1-\alpha) K^\alpha L^{-\alpha} L}{A K^\alpha L^{1-\alpha}}=1-\alpha
        $$
        \item יחס הון תוצר קבוע לאורך זמן (ב $S.S$ במידה ושום דבר לא משתנה): $$\frac{K}{Y} = \frac{\tilde{k}}{\tilde{y}} = \frac{s}{n+d+g}$$
    \end{itemize}
    

\end{frame}

\end{RTL}
\end{document}