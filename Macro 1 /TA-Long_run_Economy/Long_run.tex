% !TEX program = xelatex
\documentclass[usenames,dvipsnames]{beamer}
\usefonttheme{serif}
\usefonttheme{structuresmallcapsserif}
\usetheme{Copenhagen}
\setbeamertemplate{headline}{} % This line removes the headline template
\usepackage{xcolor}
\beamertemplatenavigationsymbolsempty
\usepackage{tikz}
\usepackage{pgfplots}
\renewcommand{\qed}{\hfill\blacksquare}
\newcommand{\D}[1]{\Delta #1}
\renewcommand{\a}{\alpha}
\usepackage{float}
\setbeamertemplate{frametitle continuation}{%
    \ifnum\insertcontinuationcount>999999999 % this command tells the program when to start counting and also the count will be in numbers and not in roman letters
    \insertcontinuationcount
    \fi}
% ============================================================ %
% HEBREW support via polyglossia %
% ============================================================ %
\usepackage{polyglossia}
\defaultfontfeatures{Mapping=tex-text, Scale=MatchLowercase}
\setdefaultlanguage{hebrew}
\setotherlanguage{english}
\newfontfamily\hebrewfont[Script=Hebrew]{Arial}
% Use \begin{hebrew} block of text \end{hebrew} for paragraphs.
% Use \texthebrew{ } and \textenglish{ } for short texts.
% ============================================================ %
\title[]{{מאקרו א' - תרגול  8 - כלכלה בטווח הארוך}}
\author{\texthebrew{ מתן לבינטוב}}
\institute[{{ אב"ג}}]{{ אוניברסיטת בן גוריון בנגב}}
\date{}
\usepackage{bidi}
\begin{document}
\begin{RTL}
\begin{frame}
\titlepage
\end{frame}
\begin{frame}
    \frametitle{נושאים}
    \tableofcontents
    

\end{frame}
\section{הנחות}
\begin{frame}[allowframebreaks]
    \frametitle{הנחות}
    קיימים 3 גורמי יצור הקובעים את כושר היצור של המשק: 
    \begin{enumerate}
        \item $L$ - כוח העבודה
        \item $K$ - מלאי ההון
        \item $A$ - רמת הטכנלוגיה
    \end{enumerate}

    \begin{block}{פונקציית היצור}
        $$Y = AF(K,L)$$
    \end{block}
    \begin{block}{הנחות}
        \begin{itemize}
            \item בטווח ארוך גורמי היצור קבועים ולכן התוצר גם קבוע - $\bar Y = \bar A F(\bar K, \bar L)$
            \item תפוקה שולית חיובית - $F_K = MPK > 0 , F_L = MPL > 0$
            \item תפוקה שולית פוחתת - $F_{KK} < 0 , F_{LL} < 0$
            \item תק''ל - תשואה קבועה לגודל
        \end{itemize}
    \end{block}
    \end{frame}

    \begin{frame}
        \frametitle{גרף יפה}
        \begin{center}
        \begin{tikzpicture}
            \begin{axis}[
                xlabel={$K$},
                ylabel={$Y$},
                axis lines=left,
                ymax=10,
                ymin=0,
                xmax=10,
                xmin=0,
                domain=0:10
            ]
            \addplot[smooth, thick,samples = 1000] {x^(0.5)}; % This is a sample function, you'll replace it with the actual function AF(K,L)
            \node at (axis cs:6,6) {$AF(K,L)$}; % Position the AF(K,L) label appropriately
            \end{axis}
        \end{tikzpicture}
        \end{center}
    
    \end{frame}

    \begin{frame}[allowframebreaks]
        \frametitle{הנחות}
        \begin{block}{תשואה לגודל}
            תשואה לגודל היא בעצם דרגת ההומוגניות של פונקציית היצור. \\ 
            עכשיו בעברית, אם נכפיל את שני גורמי היצור בקבוע, נקבל שזהה ללכפול את הפונקציה עצמה באותו קבוע בחזקרה כלשהי, אותה חזקה היא דרגת ההומוגניות של הפונקציה.
            $$Y(\lambda) = AF(\lambda K, \lambda L) = \lambda^{s} AF(K,L) = \lambda^s Y $$
        \end{block}

        \begin{enumerate}
            \item תשואה עולה לגודל (תע''ל) - $s>1 \implies Y(\lambda) > \lambda Y $ 
            \item תשואה קבועה לגודל (תק''ל) - $s = 1 \implies Y(\lambda) = \lambda Y $
            \item תשואה יורדת לגודל (תי''ל) - $s<1 \implies Y(\lambda) < \lambda Y $
        \end{enumerate}
        
        \framebreak

        \begin{exampleblock}{דוגמה לפונקציית יצור תק''ל} 
            $$Y = AF(K,L) = K^{0.5} L^{0.5}$$
        \begin{align*}
            Y(\lambda) = AF(\lambda K , \lambda L ) &= (\lambda K )^{0.5} (\lambda L ) ^{0.5} = \lambda^{0.5} \cdot K ^{0.5} \cdot  \lambda ^{0.5} \cdot L ^{0.5} \\ =\lambda \underbrace{K^{0.5} L^{0.5}}_{=Y} &= \lambda Y
        \end{align*}
            
        \end{exampleblock}
    
    \end{frame}

    \section{הפירמה }
    \begin{frame}
        \frametitle{הפירמה}
        \begin{block}{פונקציית הרווח של הפירמה}
            $$\pi = PY - WL - RK = P \cdot AF(K,L) - WL - RK $$
            $W$ - שכר נומינלי, $R$ - מחיר הון נומינלי
            הביקוש לגורמי יצור נקבע לפי בעיית האופטימיזציה שפותרת הפירמה, כלומר מיקסום הרווח.
            $$\max \pi$$

            $$\frac{\partial \pi}{\partial L } = P \cdot \frac{\partial Y}{\partial L} - W = 0 \to MPL = \frac{W}{P}$$
            $$\frac{\partial \pi}{\partial K} = P \cdot \frac{\partial Y}{\partial K} - R = 0 \to MPK = \frac{R}{P} = i_c$$

            $i_c$ = מחיר ההון הריאלי של הפרימה
        \end{block}
        
    
    \end{frame}


\end{RTL}
\end{document}