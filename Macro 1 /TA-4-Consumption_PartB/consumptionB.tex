% !TEX program = xelatex
\documentclass[usenames,dvipsnames]{beamer}
\setbeamertemplate{frametitle continuation}{%
    \ifnum\insertcontinuationcount>999999999 % this command tells the program when to start counting and also the count will be in numbers and not in roman letters
    \insertcontinuationcount
    \fi}
\usefonttheme{serif}
\usefonttheme{structuresmallcapsserif}
\usetheme{Copenhagen}
\usepackage{xcolor}
\beamertemplatenavigationsymbolsempty
\usepackage{tikz}
\usepackage{pgfplots}
\renewcommand{\qed}{\hfill\blacksquare}
\newcommand{\D}[1]{\Delta #1}
\renewcommand{\a}{\alpha}
\usepackage{float}
% ============================================================ %
% HEBREW support via polyglossia %
% ============================================================ %
\usepackage{polyglossia}
\defaultfontfeatures{Mapping=tex-text, Scale=MatchLowercase}
\setdefaultlanguage{hebrew}
\setotherlanguage{english}
\newfontfamily\hebrewfont[Script=Hebrew]{Arial}
% Use \begin{hebrew} block of text \end{hebrew} for paragraphs.
% Use \texthebrew{ } and \textenglish{ } for short texts.
% ============================================================ %
\title[]{{מאקרו א' - תרגול 5 צריכה פרטית ב'}}
\author{\texthebrew{ מתן לבינטוב}}
\institute[{{ אב"ג}}]{{ אוניברסיטת בן גוריון בנגב}}
\date{20 בנובמבר 2023}
\usepackage{bidi}
\begin{document}
\begin{RTL}
\begin{frame}
\titlepage
\end{frame}

\begin{frame}{נושאים}
    \tableofcontents
\end{frame}

\section{תאוריית מחזורי החיים של מודליאני}

\begin{frame}[allowframebreaks]{תאוריית מחזורי החיים של מודליאני}
הפרט חי , בתוחלת $L$ שנים. מתוכם הוא עובד $N$ שנים, וביתר הזמן הוא $N-L$ הוא בפנסייה.
\begin{block}{הנחות המודל}
    \begin{itemize}
        \item הפרט \textbf{יודע בוודאות} את מספר שנות חייו
        \item אין ריבית
    \end{itemize}
    על ידי שילוב הנחה שאין ריבית ובדרך כלל גם לא תהיה העדפה לתקופת צריכה, הפרט יצרך אותה כמות בכל תקופה ולכן $C$ קבוע.
\end{block}

\begin{block}{פונקציית הצריכה לטווח ארוך (לכל החיים)}
    כיוון שהפרט לא מוריש דבר ומשתמש בכך החסכונות (פנסייה) שלו, אז סך הכנסות שלו צריכות להיות שוות לסך הצריכה שלו :
    \begin{equation*}
       \underbrace{ L \cdot C }_{\text{סך הצריכה}}= \underbrace{N \cdot Y }_{\text{סך הכנסות}}
    \end{equation*}    

    לכן פונקציית הצריכה של טווח ארוך היא :
    $$
    C = \frac{N}{L}\cdot Y
    $$
\end{block}
\framebreak % the command equivalent to \newpage in beamer
משתני המודל הם :
\begin{itemize}
    \item $Y$ - הפרט מרוויח \textbf{הכנסה קבועה} עד $N$ ואז בפנסיה אין לו הכנסה יותר
    \item $W$ - העושר שלו נצבר בכל תקופות העבודה שלו מכיוון שהוא מגדיל את העושר שלו על ידי חיסכון
    \item $C$ - הצריכה שלו, לפי הנחות המודל הצריכה נגזר שהצריכה היא קבועה
\end{itemize}
\begin{flushleft}
    \begin{tikzpicture}
 % Axes
 \draw[->] (0,0) -- (6,0) node[right] {$Age$};
 \draw[->] (0,0) -- (0,6) node[left] {$C,Y,W$};

 \draw[ultra thick,dashed,blue] (0,1.5) -- (5,1.5) node[above right] {$Consumption$};
 \draw[ultra thick,dashed,blue] (5,1.5) -- (5,0);
 \fill (5,0) circle (2pt) node[below] {$L$};
 \fill (0,1.5) circle (2pt) node[left] {$C=\frac{N}{L}\cdot Y$};
 \draw[ultra thick,red] (0,3) -- (3.5,3) node[above left] {$Income$};
 \draw[ultra thick,red] (3.5,3) -- (3.5,0);
 \draw[ultra thick,red] (3.5,0) -- (5,0);

 \fill (3.5,0) circle (2pt) node[below] {$N$};
 \fill (0,3) circle (2pt) node[left] {$Y$};
 \draw[ultra thick,dotted] (0,0) -- (3.5,6) node[above] {$Wealth$};
 \draw[ultra thick,dotted] (3.5,6) -- (5,0);
 \draw[dashed] (3.5,6) -- (3.5,0);



 


 


%  \fill (0,5) circle (2pt) node[left] {$Y_1(1+r)+Y_2$}; 
%  \fill (0,2.5) circle (2pt) node[left] {$Y_2$};
%  \draw[dashed] (0,2.5) -- (2.5,2.5) node[below] {};
%  \draw[dashed] (2.5,2.5) -- (2.5,0) node[below] {};
%  \fill(2.5,0) circle (2pt) node[below] {$Y_1$};
%  \fill (5,0) circle (2pt) node[below] {$Y_1 + \frac{Y_2}{1+r}$};
%  \fill (2.5,2.5) circle (2pt) node[above right] { ההון בשוק משתתף לא };





    \end{tikzpicture}
\end{flushleft}

\framebreak
\begin{block}{פונקצית התצרוכת בטווח קצר (כל נקודה בזמן $t$)}
    שוב נזכיר שהפרט צורך את כל הכנסתו מהנקודה שהוא נמצא בה ועד סוף החיים שלו באופן אחיד
    $$
   \underbrace{ C\left(t\right) \cdot \left(L-t\right)}_{\text{סך צריכה עתידית}} = \underbrace{W}_{\text{עושר שנצבר עד כה}} + \underbrace{ \left(N-t\right) \cdot Y}_{\text{הכנסה עתידית צפויה}}
   $$
   $$
   \implies C(t) = \underbrace{\frac{1}{L-t}}_{\text{נש''צ מתוך עושר - $\alpha$}}\cdot W +\underbrace{ \frac{N-t}{L-t}}_{\text{נש''צ מתוך הכנסה - $c$}}\cdot Y
   $$


    
\end{block}

\begin{exampleblock}{מסקנות}
    \begin{itemize}
        \item הנש''צ מתוך העושר הולך \textbf{וגדל} עם $t$, כלומר ככל שהפרט מבוגר יותר ככה הוא צורך יותר מהעושר
        \item הש''צ מתוך הכנסה הולך \textbf{וקטן} עם $t$, כלומר ככל שהפרט מבוגר יותר ככה הוא צורך פחות מהכנסתו
    \end{itemize}
    
\end{exampleblock}
\end{frame}

\section{תאוריית ההכנסה הפרמננטית}
\begin{frame}[allowframebreaks]{תאוריית ההכנסה הפרמננטית}
    תאוריה זאת מניחה שהכנסה היא תנודתית ולפרט יש 2 מרכיבים להכנסה השוטפת
    \begin{itemize}
        \item הכנסה פרמנננטית ($Y^{P}$) - המרכיב היציב שבהכנסה, והפרטים צופים שהפריט הזה ישמור על ערכו על פני זמן
        \item הכנסה טרנזיטורית ($Y^T$) - המרכיב התנודתי בהכנסה , ולכן תנודות בו אינן מלמדות דבר על הכנסה עתידית. בפרט על פני זמן נצפה שתנודות בהכנסה הטרנזיטורית יקזזו אחד את השני, כלומר $\mathbb{E}(Y^T)=0$
    \end{itemize}
    על פי התאוריה הפרטים צורכים רק לפי הכנסה פרמנננטית $Y^{P}$ ולכן $C = c\cdot Y^P$ \\
     נמ''צ  $APC = \frac{C}{Y}$

     \framebreak
     בטווח קצר :
     \begin{equation*}
        APC = \frac{c\cdot Y ^P}{Y^P + Y^T}
        \begin{cases}
            APC < c \impliedby &  Y^T > 0 \\
            APC \geq c  \impliedby & Y^T \leq 0 
        \end{cases}
     \end{equation*}
     בטווח ארוך : \\
     כפי שנאמר בטווח ארוך $\mathbb{E}(Y^T) = 0$ ולכן $APC = c$

     \framebreak

    \begin{block}{איך פרטים מחלקים את הכנסתם}
        בפועל וגם אמפירית הפרטים רואים את הכנסתם השוטפת (הכוללת) ואז בהתאם למידע שיש להם הם מחלקים אותה לקבועה או זמנית / חד פעמית. \\

        כאשר קבועה הכוונה ל$Y^P$ ולזמנית ל$Y^T$. \\ 
        ישנם 2 סוגי ציפיות :
        \begin{itemize}
            \item ציפיות אדפטיביות ("הולך עם הזרם") 
            \item ציפיות רציונליות
        \end{itemize}
    \end{block}
    \framebreak
    \begin{block}{ציפיות אדפטיביות}
        כאשר הציפיות אדפטיביות, אז הפרט משקלל את ההכנסה הפרמננטית שלו על ידי ההכנסה השוטפת שלו מהתקופה הנוכחית והקודמת.
        
    \end{block}
    \[
            Y^P_t = \theta Y_t  +  (1-\theta)Y_{t-1}
            \]
            בצורה פורמלית וכללית יותר
            \[
            Y^P_t = \sum_{i=0}^{\infty} \theta^i Y_{t-i}
                \]
                \begin{alertblock}{הערה}
                    לרוב יהיו לכם רק 2 או 3 תקופות כי אין טעם לסבך את החישובים
                \end{alertblock}
    אם נציב בפונקצית הצריכה נקבל : 
    \begin{equation*}
        C_t = c\cdot Y^P_t = \underbrace{\theta\cdot c}_{\text{נש''צ}}\cdot Y_t  + \underbrace{ (1-\theta)c\cdot Y_{t-1}}_{\text{קבוע / חותך}}
    \end{equation*}
    מה משפיע על $\theta$ ?
    \begin{enumerate}
        \item סוג תעסוקה - ככל שתעסוקה יותר יציבה הפרט ייתן משקל גדול יותר להכנסה שוטפת בתקופה נוכחית וייחס חשיבות רבה לשינויים בשכר
        \item הסיבה לשינוי בהכנסה - הפרט בוחן אם האירוע שגרם לשינוי הוא חד פעמי לדוג' בונוס או אירוע קבוע / פרמנננטי כמו קידום בעבודה
    \end{enumerate}

    \framebreak
    \begin{block}{גישת הרציונליות הקיצונית - Walk Random}
        הפרט משקלל את כל השינויים \textbf{העתידים} לקרות כדי להחליט מה תהיה רמת הצריכה שלו, \textbf{ולכן רק שינויים בלתי צפויים יכולים לשנות את הצריכה שלו.}


        
    \end{block}
\end{frame}


\end{RTL}
\end{document}