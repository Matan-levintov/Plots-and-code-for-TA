% !TEX program = xelatex
\documentclass[usenames,dvipsnames]{beamer}
\usefonttheme{serif}
\usefonttheme{structuresmallcapsserif}
\usetheme{Copenhagen}
\usepackage{xcolor}
\beamertemplatenavigationsymbolsempty
\usepackage{tikz}
\usepackage{pgfplots}
\renewcommand{\qed}{\hfill\blacksquare}
\newcommand{\D}[1]{\Delta #1}
\renewcommand{\a}{\alpha}
\usepackage{float}
\setbeamertemplate{frametitle continuation}{%
    \ifnum\insertcontinuationcount>999999999 % this command tells the program when to start counting and also the count will be in numbers and not in roman letters
    \insertcontinuationcount
    \fi}
% ============================================================ %
% HEBREW support via polyglossia %
% ============================================================ %
\usepackage{polyglossia}
\defaultfontfeatures{Mapping=tex-text, Scale=MatchLowercase}
\setdefaultlanguage{hebrew}
\setotherlanguage{english}
\newfontfamily\hebrewfont[Script=Hebrew]{Arial}
% Use \begin{hebrew} block of text \end{hebrew} for paragraphs.
% Use \texthebrew{ } and \textenglish{ } for short texts.
% ============================================================ %
\title[]{{מאקרו א' - תרגול 5 - צריכה ממשלתית}}
\author{\texthebrew{ מתן לבינטוב}}
\institute[{{ אב"ג}}]{{ אוניברסיטת בן גוריון בנגב}}
\date{}
\usepackage{bidi}
\begin{document}
\begin{RTL}
\begin{frame}
\titlepage
\end{frame}
\begin{frame}{נושאים}
    \tableofcontents
\end{frame}
\section{צריכה ממשלתית}
\begin{frame}[allowframebreaks]{צריכה ממשלתית}
    ראשית נציג איך בנוי התקציב הממשלתי:
    \begin{table}
    \centering
    \begin{tabular}{|c|c|} \hline 
         הכנסות& הוצאות\\ \hline 
         סך הכנסות ממיסים ברוטו  - $T_t$& צריכה ציבורית  - $G_t$\\ \hline 
         בניכוי : \\ תשלומי העברה $TR$ \\ ריבית על חוב נצבר $rD_{t-1}$&השקעה ציבורית $I_t^G$ \\ \hline
    \end{tabular}
    \end{table}
גרעון :
\begin{equation*}
    DEF_t = \underbrace{ G_t + I_t^G}_{\text{הוצאות ממשלה}} - \underbrace{(T_t -TR_t - rD_{t-1})}_{\text{הכנסות נטו ממשלה}}
\end{equation*}

גירעון בסיסי :
\begin{equation*}
    F_t = G_t + I_t^G - (T_t -TR_t) = DEF_t - rD_{t-1}
\end{equation*}
\begin{block}{חיסכון ציבורי :
    }
    \begin{equation*}
        \underbrace{ S_G}_{\text{חיסכון ציבורי נטו}} = \underbrace{T - TR -rD_{t-1}}_{\text{הכנסות ציבוריות נטו}}\quad  -\underbrace{G}_{\text{הוצאות ציבוריות}}
     \end{equation*}    
\end{block}

מכאן נובע, 
\begin{equation*}
    DEF - I_G = -S_G
\end{equation*}
במילים, גרעון בניכוי השקעות שווה לחיסכון הממשלה בערך מוחלט.

\end{frame}
\section{החוב הציבורי}
\begin{frame}[allowframebreaks]
    \frametitle{החוב הציבורי}
    \begin{block}{משוואות}
        \begin{equation*}
            \underbrace{D_t}_{\text{מלאי}} = \underbrace{D_{t-1}}_{\text{מלאי}} + \underbrace{DEF}_{\text{זרם}}
        \end{equation*}
        \begin{equation*}
            DEF = D_t - D_{t-1}
        \end{equation*}
        \begin{equation*}
            \dot D_t = DEF_t
        \end{equation*}
    \end{block}

    נטל החוב , זה בעצם החוב ביחס לתוצר :
    \[d_t = \frac{D_t}{Y_t}\]
    שינויים על פני זמן (בזמן רציף) : 
    \[\dot D = DEF  = F + rD\]
    \begin{block}{שיעור השינוי בחוב + שיעור השינוי בנטל החוב }
    \[\widehat D = \frac{\dot D }{D } = \frac{F}{D} + r  \underbrace{\implies}_{\widehat d = \widehat D - \widehat Y} \widehat d  = \frac{F}{D} + r - \widehat{Y} \]
        
    \end{block}
\end{frame}
\section{שקילות רקרדו}
\begin{frame}[allowframebreaks]
    \frametitle{שקילות רקרדו}
    הרחבה למודל הצריכה הדו תקופתית שלמדנו בשיעורים הקודמים, כעת ההכנסה של הפרט תלויה
בגביית המיסים בזמן 1 ובזמן .2 הפרטים במודל שלנו הם פרטים רציונליים וחכמים אשר מבינים
שאם הממשלה מעלה את הצריכה שלה היום, היא תעלה מיסים או היום או בתקופה הבאה. לכן
המסקנה היא שזה לא משנה אם הממשלה תממן את הגדלת הצריכה באמצעות מיסים או לקיחת
חוב. אנחנו נראה כי כאשר מתקיימות מגבלות בשוק ההון לא נגיע בהכרח לאותן מסקנות.
    
\framebreak
\begin{alertblock}{שקילות רקרדו נשברת כאשר : }
    
\begin{enumerate}
    \item מגבלת אשראי לפרטים
    \item ריבית שונה לממשלה ופרטים
    \item ריבית שונה בין מלווים ולווים
\end{enumerate}
\end{alertblock}

\framebreak
קו התקציב החדש של הפרטים:
\[C_1 + \frac{C_2}{1+r} = (Y_1 - G_1) +\frac{(Y_2 - G_2)}{1+r} \]
תקציב הממשלה :
\[G_1 = T_1 + DEF_1\]
\[G_2  +(1+r)DEF_1 = T_2\]
בכל מקרה קו התקציב לא משתנה במידה והפרטים רציונלים בין הבחירה לעלות מיסים היום או בתקופה השנייה
\[C_1 + \frac{C_2}{1+r} = (Y_1 - G_1) +\frac{(Y_2 - G_2)}{1+r} + \Delta DEF_1 \]
\end{frame}
\end{RTL}
\end{document}