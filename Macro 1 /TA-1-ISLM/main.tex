% !TEX program = xelatex
\documentclass[a4paper,11pt]{article}
\usepackage{graphicx} % Required for inserting images
\usepackage[]{geometry} % Insert size in the [] to define and control the margins on the page
\usepackage{hyperref} % Makes links and table of contents clickable
% \hypersetup{colorlinks=true, linkcolor=blue} Use this if you want the links to be noticeable and have color blue, the color can be changed
% \usepackage{mathtools} % math tools builds upon amsmath, so only one of them is needed, note that mathtools has all the functions that amsmath has and even more
\usepackage{amsmath ,amssymb,amsthm}
\usepackage{tikz}
\usepackage{tikz-cd}
\renewcommand{\qed}{\hfill\blacksquare}
\newcommand{\D}[1]{\Delta #1}
\renewcommand{\a}{\alpha}
% \newcommand{\a}{\alpha}
\usepackage{cancel}
% ============================================================ %

% HEBREW support via polyglossia %

% ============================================================ %

\usepackage{polyglossia}

\defaultfontfeatures{Mapping=tex-text, Scale=MatchLowercase}

\setdefaultlanguage{hebrew}

\setotherlanguage{english}

\newfontfamily\hebrewfont[Script=Hebrew]{Arial}

% Use \begin{hebrew} block of text \end{hebrew} for paragraphs.

% Use \texthebrew{ } and \textenglish{ } for short texts.

% ============================================================ %

\usepackage{bidi} % we use bidi for Right-To-Left (RTL) writing
\title{מאקרו א' - תרגול 1}
\author{מתן לבינטוב}
\date{}
% Begin of Document ---------------------------------------- %
\begin{document}
\begin{RTL}
    
% Title ---------------------------------------- %
\maketitle
% \newpage
% Table of Contents ---------------------------------------- %
% \tableofcontents
% \newpage
% Start of paper ---------------------------------------- %
\section{הקדמה}
בתרגול הזה נדבר על מודל $IS-LM$  בטווח הקצר במשק סגור 
\section{הנחות המודל}
\begin{itemize}
    \item התוצר נקבע בטווח קצר לפי ביקוש
    \item בטווח קצר המחירים קשיחים
    \item המשק יכול בנקודת שיווי משקל אשר גבוהה יותר או נמוכה יותר מתוצר פוטנציאלי
\end{itemize}
\section{עקומת $IS$ - שוק המוצרים}
עקומת $IS$ הינה העקומה אשר על גביה נמצאים אוסף הצירופים של תוצר וריבית אשר מביעים לש''מ בשוק המוצרים
\begin{align*}
    \begin{split}
        IS : Y &= C + I + G \\
        C & = C_0 + cY^d = C_0  + c\left( Y-T \right)  ; \quad T = T_0 + tY \\
        I & =   I_0 - bi \\ 
        G &= G_0 \\ 
    \end{split}
\end{align*}

\begin{equation*}
    IS : Y = \alpha \left( A_0 - bi \right) \qquad \alpha = \frac{1}{1-c(1-t)} \: \text{המכפיל הקיינסיאני}
\end{equation*}
\begin{equation*}
    A_0 = C_0 - cT_o + I_0 + G_0
\end{equation*}

\subsection{שיפוע העקומה}
בגלל הצירים שיפוע העקומה צריך להיות לפי $\frac{\partial i}{\partial Y}$.
אולם יותר קל לגזור את העקומה שהיא לפי $\frac{\partial Y}{\partial i}$ ופשוט לעלות בחזקת מינוס 1
\begin{equation*}
    \frac{\partial Y}{\partial i} = - \alpha b \implies \frac{\partial i}{\partial Y } = \frac{-1}{\alpha b}
 \end{equation*}

\subsection{היסטים}
$\Delta Y = \alpha \left[\Delta A_0 - b\Delta i\right]$  \vspace{10pt}
\\
\textbf{היסט אופקי } $\Delta i = 0 $ \vspace{10pt} 
\\
$\D{Y} = \a \D{A_0}$ \vspace{10pt}
\\ 
\textbf{היסט אנכי } $\D{Y} = 0 $ \vspace{10pt}
\\
$\D{i} = \dfrac{\D{A_0}}{b}$
\end{RTL}
\begin{tikzpicture}[scale=2]
    \draw[->] (0,0) -- (6,0) node[right] {$Y$};
    \draw[->] (0,0) -- (0,5) node[above] {$i$};
    % \draw[dashed] (0,4) -- (4,0) node[midway, above left] {up} node[midway, below right] {down};
    \draw [blue] (0.5,3.5) -- (4.5,0.5) node[midway, above left] {};
    \node[right] at (3,3) {היצע עודף};
    \node[left] at (2.5,1) {ביקוש עודף};

\end{tikzpicture}
\newpage

\begin{RTL}    
\subsection{עקומת $LM$ - שוק הכסף}
עקומת LM היא עקומה אשר על גביה נמצאים אוסף הצירופים של תוצר וריבית אשר מביאים לשיווי משקל
בשוק הכסף. 
\begin{align*}
    \begin{split}
       LM : \left(\frac{M}{P}\right) ^ d  & = M ^ s \\ 
       \left(\frac{M}{P}  \right) ^ d    & = M_0 + kY - hi 
    \end{split}
\end{align*}
% \begin{tikzpicture}[scale=1.5]
%      \node (equation) at (0,0) {$y = C + I + G$};
%      \node (consumption) at (-1.5,-1) {Consumption};
%      \node (investments) at (0,-1) {Investments};
%      \node (government) at (1.5,-1) {Government};
%      \draw[->] (consumption) -- (-1.5,-0.25);
%      \draw[->] (investments) -- (0,-0.25);
%      \draw[->] (government) -- (1.5,-0.25);
% \end{tikzpicture}
% \begin{tikzcd}
%     & y \arrow[d, "C" description] \arrow[d, "I" description] \arrow[d, "G" description] \\
%     & C \arrow[r] & I \arrow[r] & G
%     \end{tikzcd}

\end{RTL}
\end{document}