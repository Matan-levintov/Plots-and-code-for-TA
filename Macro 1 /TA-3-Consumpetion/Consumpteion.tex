% !TEX program = xelatex
\documentclass[usenames,dvipsnames]{beamer}
\usefonttheme{serif}
\usefonttheme{structuresmallcapsserif}
\usetheme{Warsaw}
\usepackage{xcolor}
\beamertemplatenavigationsymbolsempty
\usepackage{tikz}
\usepackage{pgfplots}
\renewcommand{\qed}{\hfill\blacksquare}
\newcommand{\D}[1]{\Delta #1}
\renewcommand{\a}{\alpha}
\usepackage{float}
% ============================================================ %
% HEBREW support via polyglossia %
% ============================================================ %
\usepackage{polyglossia}
\defaultfontfeatures{Mapping=tex-text, Scale=MatchLowercase}
\setdefaultlanguage{hebrew}
\setotherlanguage{english}
\newfontfamily\hebrewfont[Script=Hebrew]{Arial}
% Use \begin{hebrew} block of text \end{hebrew} for paragraphs.
% Use \texthebrew{ } and \textenglish{ } for short texts.
% ============================================================ %
\title[]{{מאקרו א' - תרגול 3 צריכה פרטית}}
\author{\texthebrew{ מתן לבינטוב}}
\institute[{{ אב"ג}}]{{ אוניברסיטת בן גוריון בנגב}}
\date{}
\usepackage{bidi}
\begin{document}
\begin{RTL}
\begin{frame}
\titlepage
\end{frame}

% \begin{frame}{דרכים ליצירת קשר לשאלות ושעות קבלה}
    % \begin{block}{מייל}
    % אפשר לשלוח לי מייל \href{mailto:malev@post.bgu.ac.il}{malev@post.bgu.ac.il}
    % \end{block}
    % \begin{block}{טלפון}
    % 0524006246
        
    % \end{block}
% \end{frame}

\begin{frame}
    \frametitle{המודל הדו תקופתי}
    מודל זה הוא מודל אשר מתאר את הצריכה והחיסכון של של הפרט כפונקציה של מצב המשק
    \begin{block}{הנחות המודל}
        \begin{itemize}
            \item הפרט יודע את אורך חייו (לרוב שתי תקופות אך ניתן להרחיב את המודל ליותר)
            \item הפרט יודע את הכנסותיו בהווה ובעתיד
            \item הפרט יכול לקחת את הכנסות שלו מהעתיד (לקחת הלוואה) או להעביר הכנסות מהווה לעתיד (לחסוך)
            \item פונקצית התועלת של הפרט מורכבת מצריכה בהווה ובעתיד
        \end{itemize}
    \end{block}
    
    פרט טכני: \\ 
    אנו מניחים כי מחיר מוצר הצריכה בהווה הוא $P_{C_1} = 1$ ולכן מחיר מוצר הצריכה בעתיד הוא $P_{C_2} = \frac{1}{1+r}$, ויחס המחירים הוא , $\frac{P_{C_1}}{P_{C_2}} = 1 + r$ 

\end{frame}

\begin{frame}
    \frametitle{מודל הדו תקופתי}
    \begin{block}{קו התקציב}
        סך הצריכה המהוונת של הפרט צריכה להיות שווה לסך הכנסה המהוונת של הפרט על מנת שהוא ימקסם תועלת מצריכה
        
        $$
        C_1 + \frac{C_2}{1+r} = Y_1 + \frac{Y_2}{1+r} \quad or \quad  C_2 = Y_2 + (Y_1-C_1) \cdot (1+r)
        $$
    \end{block}
    \begin{itemize}
        \item  בגלל שאנחנו בצירים של $(C_1,C_2)$ אז שיפוע קו התצקיב הוא $1+r$ 
        \item כל שינוי בהכנסה (בין אם זה בהווה או בעתיד) יעלה את קו התצקיב למעלה או למטה
        \item שינוים בריבית ישנו את שיפוע הקו
    \end{itemize}

\end{frame}

\begin{frame}
    \frametitle{מודל דו תקופתי - ציור}
    \begin{flushleft}
        \begin{tikzpicture}
            \tikzset{
        position label/.style={
           below = 3pt,
           text height = 1.5ex,
           text depth = 1ex
        },
       brace/.style={
         decoration={brace,raise = 0.5ex},
         decorate
       }
    }
                % Axes
                \draw[->] (0,0) -- (6,0) node[right] {$C_1$};
                \draw[->] (0,0) -- (0,6) node[above] {$C_2$};

                \draw[thick, blue] (0,5) -- (5,0) node[above] {};

                \fill (0,5) circle (2pt) node[left] {$Y_1(1+r)+Y_2$}; 
                \fill (0,2.5) circle (2pt) node[left] {$Y_2$};
                \draw[dashed] (0,2.5) -- (2.5,2.5) node[below] {};
                \draw[dashed] (2.5,2.5) -- (2.5,0) node[below] {};
                \fill(2.5,0) circle (2pt) node[below] {$Y_1$};
                \fill (5,0) circle (2pt) node[below] {$Y_1 + \frac{Y_2}{1+r}$};
                \fill (2.5,2.5) circle (2pt) node[above right] { ההון בשוק משתתף לא };
                \draw (4,0)  arc(-45:0:1) node[below left] (1+r) {$1+r$};
                \draw [brace] (0,5) -- node [position label, pos=.5,yshift = 4.5ex, xshift = 5ex] {\textcolor{green!70!black}{חוסך הפרט}} (2.5,2.5);
                % \draw [decoration={brace,raise=0.5ex}, decorate] (0,5) -- (2.5,2.5);
                \draw [brace] (2.5,2.5) -- node [position label, pos=.5,yshift = 4.5ex, xshift = 5ex] {\textcolor{red}{לווה הפרט}} (5,0);

                

                
            \end{tikzpicture}
    \end{flushleft}
\end{frame}

% \begin{frame}
%     \frametitle{$S$}
%     \begin{tikzpicture}
        
%     \tikzset{
%         position label/.style={
%            below = 3pt,
%            text height = 1.5ex,
%            text depth = 1ex
%         },
%        brace/.style={
%          decoration={brace, mirror},
%          decorate
%        }
%     }
    
%     \node [position label] (cStart) at (0,0) {$ \underline{c} $};
%     \node [position label] (cA) at (2.3,0) {$ c_A^{*} $};
%     \node [position label] (cB) at (4,0) {$ c_B^{*} $};
%     \node [position label] (cEnd) at (8,0) {$\overline{c} $};
    
%     \draw [brace] (cStart.south) -- (cA.south) node [position label, pos=0.5] {First};
% \end{tikzpicture}

% \end{frame}

\begin{frame}
    \frametitle{תועלת הפרט}
    הפרט מפיק תועלת מצריכת מוצרים בהווה ובעתיד
    \begin{equation*}
        U(C_1,C_2) = u_1(C_1) + \frac{1}{\rho} \times u_2(C_2)
    \end{equation*}

    כאשר $\rho$ היא העדפת הזמן של הפרט
    \begin{itemize}
        \item $\rho > 1$ הפרט מעדיף צריכה בהווה
        \item $\rho < 1$ הפרט מעדיף צריכה בעתיד
        \item $\rho = 1$ לפרט אין העדפת זמן
    \end{itemize}    

\end{frame}

\begin{frame}
    \frametitle{פיתרון בעיית הפרט}
    \begin{equation}
        \max U(C_1,C_2) 
    \end{equation}
    \begin{equation}
        s.t \quad C_2 = Y_2 + (Y_1 - C_1) \cdot (1+r)
    \end{equation}
    \begin{equation}
        MRS = 1+r
    \end{equation}
    \begin{equation}
        C_1 = f\left(Y_1,Y_2,r\right)  \quad C_2 = f\left(Y_1,Y_2,r\right)
    \end{equation}
\end{frame}

\begin{frame}{הערות}
    \begin{block}{אפקטים ודברים חשובים}
        \begin{itemize}
            \item אפקט התחלופה - כאשר מחיר של מוצר אחד עולה אזי נרצה לצרוך ממנו פחות
            \item אפקט ההכנסה - כהכנסה שלנו עולה אני רוצה לצרוך יותר מהכל
            \item מגבלות בשוק ההון - מגבלות בשוק ההון יכולות לבוא לידי ביטוי על ידי ריבית שונה למלווים ולווים, הגבלה על כמות הלוואות
            \item לפרט שלא משתתף בשוק ההון יש נטייה שולית לצרוך והיא שווה ל1
        \end{itemize}
    \end{block}
    
    
\end{frame}
\end{RTL}
\end{document}