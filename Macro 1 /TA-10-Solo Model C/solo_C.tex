% !TEX program = xelatex
\documentclass[usenames,dvipsnames]{beamer}
\usefonttheme{serif}
\usefonttheme{structuresmallcapsserif}
\usetheme{Copenhagen}
\setbeamertemplate{headline}{} % This line removes the headline template
\usepackage{xcolor}
\beamertemplatenavigationsymbolsempty
\usepackage{tikz}
\usepackage{pgfplots}
\pgfplotsset{compat=newest}
\renewcommand{\qed}{\hfill\blacksquare}
\newcommand{\D}[1]{\Delta #1}
\renewcommand{\a}{\alpha}
\usepackage{float}
\setbeamertemplate{frametitle continuation}{%
    \ifnum\insertcontinuationcount>999999999 % this command tells the program when to start counting and also the count will be in numbers and not in roman letters
    \insertcontinuationcount
    \fi}
\definecolor{darkgreen}{rgb}{0.0, 0.5, 0.0}

% ============================================================ %
% HEBREW support via polyglossia %
% ============================================================ %
\usepackage{polyglossia}
\defaultfontfeatures{Mapping=tex-text, Scale=MatchLowercase}
\setdefaultlanguage{hebrew}
\setotherlanguage{english}
\newfontfamily\hebrewfont[Script=Hebrew]{Arial}
% Use \begin{hebrew} block of text \end{hebrew} for paragraphs.
% Use \texthebrew{ } and \textenglish{ } for short texts.
% ============================================================ %
\title{מאקרו א' - תרגול 11 צמיחה ג'}
\author{\texthebrew{ מתן לבינטוב}}
\institute[{{ אב"ג}}]{{ אוניברסיטת בן גוריון בנגב}}
\date{}
\usepackage{bidi}
\begin{document}
\begin{RTL}
\begin{frame}
\titlepage
\end{frame}

\begin{frame}
    \frametitle{נושאים}
    \tableofcontents
\end{frame}

\section{צמיחה אנדוגנית}
\begin{frame}[allowframebreaks]
    \frametitle{צמיחה אנדוגנית}
    הרמה הטכנולוגית תלויה בהון לעובד, כאשר מלאי ההון לעובד במשק עולה הרמה
    הטכנולוגית עולה. עובדה זו יכולה לגרום לכך שלהון לעובד לא בהכרח יתקיים תפוקה
    שולית פוחתת.
    \\
    נניח מקדם אנושי - $h$, כאשר $h = \mu k $
    $$Y = AF\left(K,hL\right) \implies y = \underbrace{AF\left(\mu\right)}_{\bar A} \implies y =\bar A k$$
    כעת משוואת התנועה היא : 
    $$\dot k = s\bar A k - \left(n+d\right) k \implies \left(s \bar A - (n+d)\right)\cdot k$$
    מהמשוואה החדשה הזאת אפשר להסיק כמה מסקנות מאוד חשובות
    

\end{frame}

\begin{frame}[allowframebreaks]
    \frametitle{מסקנות חשובות עם צמיחה אנדוגנית}
    \begin{itemize}
        \item המשק יכול להיות או בהצבר הון חיובי ותמיד לצמוח, כלומר לא יהיה מצב עמיד או בהצבר הון שלילי ואז לקטון עד כדי קריסה $k=0$
        \begin{tikzpicture}[scale=0.95]
            \begin{axis}[
                axis lines = left,
                xlabel = $k$,
                ylabel = $y$,
                xmin=0, xmax=14,
                ymin=0, ymax=14,
                legend pos=north west,
            ]
            
            
            % Add the lines
            \addplot [domain=0:12, samples=200, color=black] {x} node[above] {$y = A k$};
            \addplot [domain=0:12, samples=200, color=darkgreen] {0.6*x} node[above]{$sy = s A k$};
            \addplot [domain=0:12, samples=200, color=red] {0.3*x} node[above] {$(n + d)k$};
            
            \end{axis}
            \end{tikzpicture}
        

        \framebreak

        \item שיעור צמיחת ההון לעובד, תלוי בשיעור החיסכון : $$\widehat k = \frac{\dot k}{k} = \frac{\left(s \bar A - (n+d)\right) \cdot k}{k} = s \bar A - (n+d) $$
        \item שיעור צמיחת התוצר לעובד שווה לשיעור צמיחת ההון לעובד : $$\widehat y = \frac{MPK \cdot \dot k}{y} = \frac{\bar A \cdot \dot k }{\bar A \cdot k} = \frac{\dot k}{k} = \widehat k $$
        
    \end{itemize}


\end{frame}


\section{חשבונאות צמיחה}
\begin{frame}[allowframebreaks]
    \frametitle{חשבונאות צמיחה}
    \begin{block}{פונקציית יצור תק''ל}
        פונקציית יצור כגון $Y = A K^\alpha L^{1-\alpha}$ מקיימת תק''ל ולכן :
        $$Y = MPK \cdot K + MPL \cdot L$$
        נזכור ש : 
        \begin{enumerate}
            \item $MPK = i_c = \alpha k $
            \item $MPL = \frac{W}{P} =  (1-\alpha) y $
        \end{enumerate}

        נהפוך את הפונקציה למושגי שיעורי השינוי (או הדפרנציאל השלם) :
        $$\widehat Y = \widehat A + S_k \cdot \widehat K + S_L \cdot  \widehat L$$
        כאשר $S_K$ היא הגמישות החלקית של ההון בתוצר ו$S_L$ היא הגמישות החלקית של העבודה בתוצר.
    \end{block}

    \framebreak
    \begin{block}{קוב דאגלס}
        בקוב דאגלס \textbf{תמיד} :
        $$S_L = (1-\alpha), \quad S_K = \alpha$$
    \end{block}

    \begin{block}{הפיריון הכולל}
        $A$ הוא הפיריון הכולל, לפעמים גם קרוי השארית של סולו משום שהוא תמיד מחושב בתור שארית.
        $$
\widehat{A}=\widehat{Y}-\alpha \widehat{K}-(1-\alpha) \widehat{L}
$$
    \end{block}
    

\end{frame}


\end{RTL}
\end{document}